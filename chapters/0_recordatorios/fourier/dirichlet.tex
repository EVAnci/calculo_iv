\section{Condiciones de Dirichlet}

La convergencia de una serie de Fourier involucra tres aspectos distintos y fundamentales:
\begin{enumerate}
  \item Que existan los coeficientes genéricos $a_n$, $b_n$ y puedan hallarse (sino, no habría convergencia ni serie infinita).
  \item Que exista la serie (además de la existencia de $a_n$ y $b_n$, se requiere que la función pueda expresarse como una serie infinita).
  \item Que la función desarrollada en serie de Fourier pueda aproximarse por un número finito de términos (se pueden despreciar el resto de términos a partir de cierto ``$n$'' y sin embargo, la expresión se aproxime lo suficiente a la original).
\end{enumerate}

\subsection{Condición débil}

Debe cumplirse para que la serie sea convergente, pero que se cumpla no nos garantiza que la serie converge (es una condición necesaria pero no suficiente).

Es decir, sabemos que si esta condición no se cumple es divergente o la serie no existe. Por tanto, debe cumplirse. Sin embargo que se cumpla no nos garantiza que la serie converja, entonces nada podemos decir respecto de su convergencia.

La condición débil reside en la existencia de los coeficientes. Como los coeficientes vienen definidos a través de una integral, la existencia de los coeficientes implica que la integral sea finita:
\begin{align*}
  a_n&=\frac{2}{T}\int_{-T/2}^{T/2}f(t)\cos\left(\frac{2n\pi}{T}t\right)\,dt<\infty\\
  b_n&=\frac{2}{T}\int_{-T/2}^{T/2}f(t)\sin\left(\frac{2n\pi}{T}t\right)\,dt<\infty
\end{align*}
Aquí $T$ es una constante (el período) y las funciones coseno y seno están acotadas entre $-1$ y $1$. Por tanto, solo deberá ser finita la integral del valor absoluto de $f(t)$. Así:
$$
\int_{-T/2}^{T/2}\lvert f(t)\rvert \,dt <\infty
$$
Ya que el coseno (o el seno) sólo afectaría a $f$ en un factor menor o igual a 1, o cambiaría de signo.

Una integral más visual y sencilla es la que se muestra a continuación, que cumple con el objetivo de prescindir del signo 
$$
\int_{-T/2}^{T/2}[f(t)]^2\,dt<\infty
$$

\subsection{Condición fuerte}

Esta condición incorpora las restricciones adicionales, que la hacen necesaria y suficiente. Para que una serie de Fourier sea uniformemente convergente, la función desarrollada $f(x)$ debe permanecer finita y tener un número finito de máximos y mínimos.

La condición débil y la condición fuerte son llamadas condiciones de Dirichlet\footnote{Johann Peter Gustav Lejeune Dirichlet (1805-1859) fue un matemático alemán.} que garantizan que la serie de Fourier de una función periódica converja puntualmente a la función original (excepto en sus puntos de discontinuidad). Estas condiciones son:
\begin{enumerate}
  \item La función debe ser absolutamente integrable en un periodo (la condición débil):
    $$
    \int_{-T/2}^{T/2} |f(t)| \, dt < \infty
    $$
    Esto asegura que los coeficientes de Fourier existan y puedan calcularse.
  \item La función debe tener un número finito de discontinuidades en un periodo: No puede tener infinitas discontinuidades en un intervalo finito.
  \item La función debe tener un número finito de máximos y mínimos en un periodo: No puede oscilar infinitamente.
\end{enumerate}

Si una función satisface estas condiciones, entonces su serie de Fourier converge a $f(t)$ en los puntos donde $f$ es continua, y al promedio de los límites izquierdo y derecho en los puntos de discontinuidad.

Con esto, hemos finalizado el capítulo de revisión de Fourier.
