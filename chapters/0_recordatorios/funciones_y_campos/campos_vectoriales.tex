\section{Campos vectoriales}

Un campo vectorial sobre $D^2 \subset \mathbb{R}^2$ (subconjunto de $\mathbb{R}^2$) es una función $\vec F$ que a cada punto $(x,y)\in D^2$ le asigna un único vector en $\mathbb{R}^2$ (esta definición también se extiende a  $\mathbb{R}^3$).

En el plano, podemos definir a un campo vectorial de la siguiente manera:
\begin{align*}
&\vec F :D^2\mapsto \mathbb{R}^2 \\
&(x,y)\rightarrow F(x,y)=M(x,y)\hat{\imath} + N(x,y)\hat{\jmath}
\end{align*}
Donde $F$ es la letra asignada al campo vectorial y $M$ y $N$ son campos escalares.

Decimos que $F$ será continuo si $M$ y $N$ lo son y $F$ será diferenciable si $M$ y $N$ lo son. Para $\mathbb{R}^3$ incluimos otro campo escalar (llámese $P$).

\subsection{Derivadas de un campo vectorial}

Asociados a las derivadas de un campo vectorial, hay dos campos, uno escalar y otro vectorial. Estos son:
\begin{itemize}
\item Divergencia
\item Rotacional
\end{itemize}

\subsection{Divergencia}

Sea $F(x,y,z)$ un campo vectorial definido en $\mathbb{R}^3$, para el que existen $\frac{\partial M}{\partial x},\frac{\partial N}{\partial y}$ y $\frac{\partial P}{\partial z}$ 
Entonces
$$
\text{div }F = \nabla \cdot F = \frac{\partial M}{\partial x} + \frac{\partial N}{\partial y} + \frac{\partial P}{\partial z}
$$
donde $\nabla$ es el operador nabla (o laplaciano) que equivale a $\vec \nabla = \left(\frac{\partial }{\partial x},\frac{\partial }{\partial y},\frac{\partial }{\partial z} \right)$ para $\mathbb{R}^3$ (para $\mathbb{R}^2$ no se tiene la última componente).

La divergencia nos dice si, en determinada zona, el campo vectorial entra o sale. Si sale $\text{div }F>0$; si entra $\text{div }F<0$.

\subsection{Rotacional}

Se define en $\mathbb{R}^3$ como:
$$
\text{rot }F = \nabla \times F = \left( \frac{\partial P}{\partial y}-\frac{\partial N}{\partial z}, \frac{\partial M}{\partial z}-\frac{\partial P}{\partial x}, \frac{\partial N}{\partial x}-\frac{\partial M}{\partial y} \right)
$$

El rotor nos dice si en un punto, el campo vectorial tiende a arremolinarse, y si lo hace de forma horaria o antihoraria.
\begin{itemize}
  \item Componente $k>0 \rightarrow$ antihorario,
  \item Componente $k<0\rightarrow$ horario.
\end{itemize}

Podemos construir el vector $\text{rot }F$ usando la forma de calculo del producto vectorial:
$$
\nabla \times F = \begin{vmatrix}
\hat{\imath} & \hat{\jmath} & \hat k \\
\frac{\partial}{\partial x} & \frac{\partial}{\partial y} & \frac{\partial}{\partial z} \\
M & N & P
\end{vmatrix}
$$

\subsection{Campo conservativo}

Un campo vectorial $F$ es conservativo si existe una función escalar $f$ tal que, su gradiente $\nabla f = F$. Es decir $F$ es el gradiente de alguna función escalar. Si $f$ existe, se le llama función potencial de $F$.

Recordemos: El gradiente de $f$ es el operador nabla aplicado a el campo escalar $f$. Es decir:
$$
\nabla f(x_1,x_2,\cdots,x_n) = \left(\frac{\partial f}{\partial x_1},\frac{\partial f}{\partial x_2},\cdots,\frac{\partial f}{\partial x_n}\right)
$$

\subsubsection*{Para $\mathbb{R}^2$}

Demostraremos que, si $F$ es conservativo, entonces 
$$
\frac{\partial N}{\partial x} = \frac{\partial M}{\partial y}
$$

Partimos suponiendo que $F=M\hat{\imath} + N\hat{\jmath}$ es conservativo, lo que quiere decir que
$$
\exists f \quad \text{tal que} \quad F = \nabla f = \frac{\partial f}{\partial x}\hat{\imath} + \frac{\partial f}{\partial y} \hat{\jmath} 
$$
Por ende
$$
M=\frac{\partial f}{\partial x} \qquad N=\frac{\partial f}{\partial y}
$$
Como la condición establece continuidad de las derivadas y como hemos partido de que el campo es conservativo, entonces es suave y diferenciable. Por lo tanto:
\begin{itemize}
\item Derivamos $M$ respecto de $y$: $\displaystyle \frac{\partial M}{\partial y} = \frac{\partial^2 f}{\partial y \partial x}$
\item Derivamos $N$ respecto de $x$: $\displaystyle \frac{\partial N}{\partial x} = \frac{\partial^2 f}{\partial x \partial y}$
\end{itemize}
Y por el teorema de derivadas cruzadas $N_x = M_y$.

Con esto hemos demostrado que para que $\vec F$ sea conservativo necesita derivadas cruzadas iguales.

Que necesite derivadas cruzadas iguales entre $M$ y $N$ es necesario, pero esto no nos garantiza que sea conservativo. Además debe cumplirse que:
$$
\oint_c M\,dx + N \,dy = 0
$$
porque, por el Teorema de Green:
$$
\oint M\,dx + N\,dy = \iint_D \frac{\partial N}{\partial x}-\frac{\partial M}{\partial y}dA
$$
Como $\frac{\partial N}{\partial x}=\frac{\partial M}{\partial y}$, entonces se tiene que:
$$
\iint_D \frac{\partial N}{\partial x}-\frac{\partial M}{\partial y}dA = 0
$$
Y, por tanto, la integral de linea sobre una curva cerrada también será cero. Luego, $F$ es conservativo.
