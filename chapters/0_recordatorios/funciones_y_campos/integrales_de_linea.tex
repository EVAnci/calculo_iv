\section{Integrales de línea}

Las integral de línea no es más que una generalización de la integral usual en funciones escalares de variable real, donde el camino de integración es un segmento rectilíneo.

Para poder hablar de integrales de línea primero necesitamos saber conceptualmente qué es lo que buscamos. Lo que queremos hacer es integrar el recorrido a través de una curva (que llamaremos $C$) por un campo vectorial. 

El concepto detrás de la integral de línea recae sobre analizar el efecto que sufre una partícula tras moverse por un campo de fuerzas. Por ejemplo, al mover una pelota sobre un campo gravitatorio ¿Cómo influye el campo gravitatorio sobre la pelota? O, si tenemos un campo vectorial que nos permite conocer la velocidad del viento en un punto del espacio ¿Cómo le afecta el viento a la pelota cuando se mueve? Son las preguntas que buscamos responder.

La integral nos permite responder a las preguntas anteriores sumando todas las pequeñas contribuciones de fuerza que ha realizado el campo sobre la partícula.
\begin{figure}[ht]
\begin{tikzpicture}[
    % --- Configuración de la vista 3D ---
    x={(1cm,0.5cm)},  % Coordenada X proyectada (perspectiva)
    y={(0.9cm,-0.4cm)},% Coordenada Y proyectada
    z={(0cm,1cm)},    % Coordenada Z vertical
    scale=1.5,
    axis_color/.style={gray, thin},
    curve_color/.style={blue, thick},
    surface_color/.style={fill=blue!10, draw=blue!50, thin}, % Estilo de la "hoja"
    func_color/.style={red, thick, smooth},
    >=Stealth           % Estilo de las flechas
]

    % --- Definir la curva paramétrica y el campo escalar ---
    % Curva C: Un cuarto de círculo en el plano XY (radio R)
    \def\R{2} % Radio de la curva
    % Campo escalar f(x,y) = x^2 + y^2 (para simplificar, haremos z = f(x(t), y(t)))
    % Para un punto (x,y) en la curva, la altura z = x^2 + y^2

    % --- Ejes ---
    \draw[axis_color, ->] (0,0,0) -- (3,0,0) node[below left] {$x$};
    \draw[axis_color, ->] (0,0,0) -- (0,3,0) node[below right] {$y$};
    \draw[axis_color, ->] (0,0,0) -- (0,0,3.5) node[above] {$z$};
    \node at (0,0,0) [below left] {$O$};

    % --- Dibujar la curva C en el plano XY (z=0) ---
    % Parametrización para el cuarto de círculo: x=R*cos(t), y=R*sin(t)
    % t de 0 a 90 grados
    \draw[curve_color, postaction={decorate, decoration={markings, mark=at position 0.55 with {\arrow{>}}}}]
        plot[domain=0:90, samples=50] (2*cos(\x), 2*sin(\x), 0) node[pos=0.8, below right] {$C$};

    % --- Dibujar la "hoja" de la integral (la superficie) ---
    % Esto lo haremos dibujando líneas verticales desde la curva C hasta z=f(x,y)
    % Y luego una "curva superior" z = f(x(t),y(t))
    
    % Dibujar líneas verticales (para dar efecto de superficie)
    % Incrementos cada 15 grados en el parámetro de la curva
    \foreach \t in {0,15,...,90} {
        \pgfmathsetmacro{\x_val}{\R*cos(\t)}
        \pgfmathsetmacro{\y_val}{\R*sin(\t)}
        \pgfmathsetmacro{\z_val}{\x_val*\x_val + \y_val*\y_val} % z = x^2 + y^2
        \draw[blue!30, thin, densely dotted] (\x_val, \y_val, 0) -- (\x_val, \y_val, \z_val);
    }

    % --- Dibuja la "tapa superior" de la hoja (la curva f(x(t),y(t))) ---
    \draw[func_color, postaction={decorate, decoration={markings, mark=at position 0.55 with {\arrow{>}}}}]
        plot[domain=0:90, samples=50] ({ \R*cos(\x) }, { \R*sin(\x) }, { \R*\R } ); % z = R^2 para f(x,y)=x^2+y^2 sobre un círculo
                                                                                  % Ojo: para otros f(x,y) esto cambia.
                                                                                  % En este caso f(x,y)=x^2+y^2 = (Rcos t)^2 + (Rsin t)^2 = R^2
                                                                                  
    % Etiqueta para el campo escalar
    \node[red, above right] at (2*cos(60), 2*sin(60), 2*2) {$f(x,y)=x^2+y^2$};

    % --- Rellenar la superficie (la "hoja") ---
    % Dibujamos un polígono uniendo los puntos de la curva base y la curva superior
    % Esto es un poco más avanzado, lo hacemos con un plot y luego rellenamos
    \fill[surface_color]
        plot[domain=90:0, samples=20] ({ \R*cos(\x) }, { \R*sin(\x) }, { \R*\R } ) -- % Parte superior (inversa)
        plot[domain=0:90, samples=20] (\R*cos(\x), \R*sin(\x), 0) -- cycle;           % Parte inferior y cierre

\end{tikzpicture}
\end{figure}
