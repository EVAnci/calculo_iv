\section{Integrales de línea}

Como se explica en \textcite{openstax_line_integral}, hay dos tipos de integrales de línea: integrales de línea escalares e integrales de línea vectoriales. Las integrales de línea escalares son integrales de una función escalar sobre una curva en un plano o en el espacio. Las integrales de líneas vectoriales son integrales de un campo vectorial sobre una curva en un plano o en el espacio.

\subsection{Integral de línea escalar}

En cálculo de una variable, una integral de Riemann
$$
\int_a^b g(x)\,dx
$$
se construye sumando ``rectángulos'' $g(x_i^\ast)\Delta x$. En el caso de una integral de línea escalar, la idea es la misma, pero en lugar de intervalos en la recta, tenemos trozos de una curva en el espacio.

Sea una curva $C$ parametrizada por:
$$
\mathbf{r}(t) = (x(t), y(t)), \quad a \leq t \leq b.
$$
donde $t$ es el parámetro (como el ``tiempo''). A medida que $t$ cambia de $a$ a $b$, recorremos la curva.

Queremos integrar una función escalar $f(x,y)$ a lo largo de la curva $C$. Primero dividimos la curva en pedacitos pequeños de longitud $\Delta s_i$ (ver figura \ref{fig:curva_c_integral}). En cada pedacito, elegimos un punto cualquiera, llamado $P_i^\ast$. Evaluamos $f(P_i^\ast)$. Multiplicamos por la longitud del pedacito:
$$
f(P_i^\ast)\,\Delta s_i.
$$
La suma de todos esos productos
$$
\sum_{i=1}^n f(P_i^\ast)\Delta s_i
$$
se parece muchísimo a una suma de Riemann.

Cuando los pedacitos se hacen infinitamente pequeños ($\Delta s_i \to 0$), esa suma converge y se define la integral de línea escalar.
\begin{figure}[ht]
  \centering
  \begin{tikzpicture}[
    % Definimos un estilo para los puntos, para no repetir código
    dot/.style={circle, fill=black, inner sep=1.5pt},
    dotast/.style={circle, fill=teal, inner sep=1.5pt}
  ]
    \draw[gray,->] (-1,0) -- (10,0) node[right] {$x$};
    \draw[gray,->] (-0.5,-0.5) -- (-0.5,4) node[above] {$y$};

    \draw[thick, red,
      decoration={
        markings,
        mark=at position 0 with {
          \node (A) [dot, label=right:\footnotesize$a$] {};
        },
        mark=at position 0.03 with {
          \node (PAast) [dotast, label=right:\scriptsize$\color{teal}P_1^\ast$] {};
        },
        mark=at position 0.06 with {
          \node (PA) [dot, label=right:\scriptsize$\color{orange!70!black}P_1$] {};
        },
        mark=at position 0.35 with {
          \node (P1) [dot, label=below:\scriptsize$\color{orange!70!black}P_{i-1}$] {}; 
        },
        mark=at position 0.4 with {
          \node (Pstar1) [dotast, label=below:\scriptsize$\color{teal}P_{i}^*$] {};
        },
        mark=at position 0.45 with {
          \node (P2) [dot, label=below:\scriptsize$\color{orange!70!black}P_i$] {};
        },
        mark=at position 0.5 with {
          \node (Pstar2) [dotast, label=below:\scriptsize$\color{teal}P_{i+1}^*$] {};
        },
        mark=at position 0.55 with {
          \node (P3) [dot, label=below:\scriptsize$\color{orange!70!black}\quad P_{i+1}$] {};
        },
        mark=at position 0.93 with {
          \node (PB) [dot, label=right:\scriptsize$\color{orange!70!black}P_n$] {};
        },
        mark=at position 0.96 with {
          \node (PBast) [dotast, label=right:\scriptsize$\color{teal}P_n^\ast$] {};
        },
        mark=at position 0.999 with {
          \node (B) [dot, label=right:\footnotesize$b$] {};
        }
      },
      postaction={decorate} % Aplica la decoración DESPUÉS de dibujar la curva
    ] 
    (0.5,1) to [out=80,in=90+55] (5,1.5) 
    to [out=-35,in=260] (9,3) node[above left] {C};

    % --- Definir coordenadas primero ---
    \path ($(A)-(0.3,0)$) coordinate (A_up);
    \path ($(PA)-(0.3,0)$) coordinate (PA_up);
    \path ($(P1)+(0,0.3)$) coordinate (P1_up);
    \path ($(P2)+(0,0.3)$) coordinate (P2_up);
    \path ($(P3)+(0,0.3)$) coordinate (P3_up);
    \path ($(PB)-(0.3,0)$) coordinate (PB_up);
    \path ($(B)-(0.3,0)$) coordinate (B_up);

    % --- Dibujar segmentos ---
    \draw[blue, thick] (A_up) -- (PA_up)
      node[midway, left] {\scriptsize$\Delta S_1$};
    \draw[blue, thick] (P1_up) -- (P2_up)
      node[midway, above, xshift=2mm] {\scriptsize$\Delta S_i$};
    \draw[blue, thick] (P2_up) -- (P3_up)
      node[midway, above, xshift=2mm] {\scriptsize$\Delta S_{i+1}$};
    \draw[blue, thick] (B_up) -- (PB_up)
      node[midway, left] {\scriptsize$\Delta S_n$};

    % --- Dibujar líneas conectoras ---
    \draw[blue] (A) -- ($(A_up)-(0.1,0)$);
    \draw[blue] (PA) -- ($(PA_up)-(+0.1,0)$);
    \draw[blue] (P1) -- ($(P1_up)+(0,0.1)$);
    \draw[blue] (P2) -- ($(P2_up)+(0,0.1)$);
    \draw[blue] (P3) -- ($(P3_up)+(0,0.1)$);
    \draw[blue] (PB) -- ($(PB_up)-(0.1,0)$);
    \draw[blue] (B) -- ($(B_up)-(0.1,0)$);
    
  \end{tikzpicture}
  \caption{Curva paramétrica en el plano.}
  \label{fig:curva_c_integral}
\end{figure}


La integral de línea escalar de $f$ a lo largo de la curva $C$ es:
$$
\int_C f(x,y)\, ds.
$$
Aquí $ds$ es el elemento diferencial de longitud de arco. En términos de la parametrización,
$$
ds = \lvert\mathbf{r}'(t)\rvert\,dt = \sqrt{\Big(\frac{dx}{dt}\Big)^2 + \Big(\frac{dy}{dt}\Big)^2}\,dt.
$$
Por lo tanto:
$$
\int_C f(x,y)\, ds \;=\; \int_a^b f(x(t), y(t)) \,\|\mathbf{r}'(t)\|\,dt.
$$

Una integral de línea escalar es la generalización de la integral de Riemann al caso en que el dominio de integración no es un intervalo recto, sino una curva en el espacio. Se construye sumando valores de $f$ en la curva, multiplicados por trozos de longitud de arco, y tomando el límite para cada trozo tendiendo a cero. Para una intuición gráfica se puede consultar el siguiente \href{https://www.youtube.com/watch?v=wCphv9dCswg}{\texttt{vídeo de YouTube}}.

\subsection{Integral de línea vectorial}

Sea un campo vectorial en el plano
$$
\vec{F}(x,y) = M(x,y)\,\hat{\imath} + N(x,y)\,\hat{\jmath},
$$
y sea $C$ una curva suave (continuamente derivable) parametrizada por un vector de posición
$$
\vec{r}(t) = x(t)\,\hat{\imath} + y(t)\,\hat{\jmath}, \quad \text{con } t \in [a,b].
$$
La integral de línea de $\vec{F}$ a lo largo de $C$ representa, por ejemplo, el trabajo realizado por el campo $\vec{F}$ al mover una partícula a lo largo de la curva $C$. Se define como:
$$
\int_C \vec{F}\cdot d\vec{r}
$$
Para evaluar esta integral, debemos expresar todos los componentes en términos del parámetro $t$.

\begin{enumerate}
    \item \textbf{El campo} $\vec{F}$ se evalúa a lo largo de la curva $C$, por lo que sustituimos $x=x(t)$ y $y=y(t)$:
    $$
    \vec{F}(\vec{r}(t)) = M(x(t),y(t))\,\hat{\imath} + N(x(t),y(t))\,\hat{\jmath}.
    $$

    \item \textbf{El diferencial de trayectoria} $d\vec{r}$ representa un cambio infinitesimal en la posición a lo largo de la curva. Lo obtenemos derivando la parametrización $\vec{r}(t)$ respecto a $t$:
    $$
    d\vec{r} = \frac{d\vec{r}}{dt}\,dt = \vec{r}\,'(t)\,dt = \left( \frac{dx}{dt}\,\hat{\imath} + \frac{dy}{dt}\,\hat{\jmath} \right)\,dt = \big( x'(t)\,\hat{\imath} + y'(t)\,\hat{\jmath} \big)\,dt.
    $$
\end{enumerate}

Ahora, sustituimos ambos en la integral, calculando el producto punto:
\begin{align*}
\int_C \vec{F}\cdot d\vec{r} &= \int_a^b \vec{F}(\vec{r}(t)) \cdot \vec{r}\,'(t)\,dt \\
&= \int_a^b \Big( M(x(t),y(t))\,\hat{\imath} + N(x(t),y(t))\,\hat{\jmath} \Big) \cdot \Big( x'(t)\,\hat{\imath} + y'(t)\,\hat{\jmath} \Big)\,dt
\end{align*}
Resolviendo el producto punto $(\hat{\imath}\cdot\hat{\imath}=1, \hat{\jmath}\cdot\hat{\jmath}=1, \hat{\imath}\cdot\hat{\jmath}=0)$, llegamos a la fórmula computacional:
$$
\int_C \vec{F}\cdot d\vec{r} \;=\; \int_a^b \Big( M(x(t),y(t))\,x'(t) + N(x(t),y(t))\,y'(t) \Big)\,dt.
$$

\textbf{Notación diferencial alternativa:}
Si utilizamos la notación de diferenciales $dx = x'(t)\,dt$ y $dy = y'(t)\,dt$, podemos reescribir la integral paramétrica de forma compacta. 
$$
\int_C \vec{F}\cdot d\vec{r} \;=\; \int_C M(x,y)\,dx + N(x,y)\,dy.
$$
Ambas expresiones son equivalentes, pero la forma paramétrica es la que se utiliza para el cálculo directo.

Si el campo $\vec{F}$ es conservativo, es decir, existe un potencial $\nabla f(x,y)$ tal que
$$
\nabla f(x,y) = \vec{F}(x,y),
$$
entonces:
$$
\int_C \vec{F}\cdot d\vec{r} \;=\; f(\vec{r}(b)) - f(\vec{r}(a)).
$$
Es decir, la integral de línea solo depende de los extremos de la curva, y no de la trayectoria.

\subsubsection{Relación con el Teorema de Green}

Si la curva $C$ es cerrada (recorre el borde de una región $D$ en sentido antihorario), el teorema de Green nos da:
$$
\oint_C \vec{F}\cdot d\vec{r} \;=\; \iint_D \left( \frac{\partial N}{\partial x} - \frac{\partial M}{\partial y} \right)\, dA.
$$
Si $\vec{F}$ es conservativo, entonces se cumple
$$
\frac{\partial N}{\partial x} = \frac{\partial M}{\partial y},
$$
lo que implica que la integral de línea cerrada es cero:
$$
\oint_C \vec{F}\cdot d\vec{r} = 0.
$$
Esto conecta la idea de conservatividad con el teorema de Green. También, puedes obtener mejor intuición visual de la integral de línea vectorial a través del siguiente \href{https://youtu.be/eNUwFNg2yEQ?si=v1UCN24Bqt7mTakh}{\texttt{video de YouTube}}.

\section{El teorema de Green}

El teorema de Green conecta dos tipos de integrales. Relaciona la integral de línea de un campo vectorial alrededor de una curva cerrada simple $C$ (que es el borde de una región) con una integral doble sobre la región plana $D$ que esa curva encierra.:w

