\begin{figure}[ht]
  \centering
  \begin{subfigure}{0.48\textwidth}
    \centering
    \begin{tikzpicture}[scale=0.7, remember picture]
      % Ejes
      \draw[->] (0,0) -- (0,5) node[above left] {$\Im(z)$};
      \draw[->] (0,0) -- (5,0) node[below right] {$\Re(z)$};

      % Vecindad en el dominio
      \filldraw[blue] (2.5,2.5) circle (2pt) node[below right] {$z_0$};
      \draw[->] (2.5,2.5)--++(200:2) node[midway, above] {$\delta$};
      \draw[red, dashed] (2.5,2.5) circle (2);
      \filldraw (2.2,3.5) circle (2pt) node[left] {$z$};
      \coordinate (zpoint) at (2.2,3.5);
    \end{tikzpicture}
    \caption{Vecindad de $z_0$ en el dominio.}
  \end{subfigure}
  \hfill
  \begin{subfigure}{0.48\textwidth}
    \centering
    \begin{tikzpicture}[scale=0.7, remember picture]
      % Ejes
      \draw[->] (0,0) -- (0,5) node[above left] {$v$};
      \draw[->] (0,0) -- (6,0) node[below right] {$u$};

      % Vecindad en la imagen
      \filldraw[teal] (2.5,1.5) circle (2pt) node[below right] {$L$};
      \draw[->] (2.5,1.5)--++(190:2.4) node[midway, above] {$\varepsilon$};
      \draw[red, dashed] (2.5,1.5) circle (2.4);
      \filldraw (3.5,1.7) circle (2pt) node[right] {$f(z)$};
      \coordinate (fpoint) at (3.5,1.7);
    \end{tikzpicture}
    \caption{Vecindad de $L=f(z_0)$ en el codominio.}
  \end{subfigure}

  \begin{tikzpicture}[remember picture, overlay]
    \draw[->, dashed, gray!60!black]
      ($(zpoint)$)
      .. controls ($(zpoint)+(3,0.3)$) ..
      ($(fpoint)$)
      node[midway, above, sloped=false] {$f$};
  \end{tikzpicture}
  \caption{Relación entre la vecindad en el dominio y su imagen por $f$.}
  \label{fig:limite_complejo}
\end{figure}
