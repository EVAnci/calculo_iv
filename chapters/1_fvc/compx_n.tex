\section[Números complejos]{Revisión: Números complejos}

Para entender funciones de variable compleja, demás está decir que es necesario entender bien los números complejos y sus operaciones. 

Para ello, en esta sección, vamos a dar una revisión o repaso a los números complejos.

\begin{definition}
  Un número complejo es un par ordenado \(a,b\) de números reales. Al conjunto de todos los números complejos se lo denomina $\mathbb{C}$, y por su definición se puede observar que $\mathbb{C}=\mathbb{R}\times\mathbb{R}$, es decir 
  $$
  \mathbb{C}=\{(a,b)|a,b \in \mathbb{R}\}
  $$
  Las propiedades de la suma, producto y potencias de números reales, serán válidas también en los números complejos.
\end{definition}

Un número complejo $z$ puede escribirse de varias formas:
\begin{itemize}
  \item Par ordenado: $z=(a,b)$
  \item Binómica: $z=a+bj$
  \item Polar: $z=(r,\theta)$
  \item Trigonométrica: $z=r(\cos(\theta)+j\sin(\theta))$
  \item Exponencial: $z=re^{j\theta}$
\end{itemize}

\subsection{Operaciones elementales}

Para un par de números complejos $z_1=(a,b)$ y $z_2=(c,d)$ en forma de par ordenado o forma binómica se definen las operaciones suma $(+)$ y producto $(\cdot)$ de la siguiente manera:
\begin{align*}
  &+:\mathbb{C}\times\mathbb{C}\to \mathbb{C} \\
  &(z_1,z_2)\rightarrow (a+c,b+d)
\end{align*}
Es decir, la suma de dos números complejos es la suma de sus componentes reales y la suma de sus componentes imaginarias. Veamos un ejemplo.
\begin{example}
  Dados $z_1=(2,4)$ y $z_2=(5,-2)$ realizamos su suma de la siguiente manera:
  \begin{align*}
    &z_1+z_2 = (2,4) + (5,-2) = (2+5,4-2) \\
    &\boxed{z_1+z_2 = (7,2)}
  \end{align*}
  De la misma manera, si expresamos en forma binómica $z_1=2+4j$ y $z_2=5-2j$ obtenemos\footnote{Aquí $j$ es la unidad imaginaria.}
  \begin{align*}
    &z_1+z_2 = (2+4) + (5-2)j = (2+5) + (4-2) \\
    &\boxed{z_1+z_2 = 7+2j}
  \end{align*}
\end{example}

El producto se define intuitivamente a partir de la forma binómica. Dados $z_1=a+bj$ y $z_2=c+dj$,
\begin{align*}
  &(\cdot) : \mathbb{C}\times\mathbb{C}\to\mathbb{C}\\ 
  &(z_1,z_2)\rightarrow (ac-bd) + (ad+bc)j
\end{align*}
Esto se obtiene tras aplicar la propiedad distributiva de la siguiente manera:
\begin{equation*}
  (a+bj)\cdot(c+dj) = ac + ad\,j + cb\,j + bd\,j^2
\end{equation*}
como $j^2=-1$ por definición, resulta
\begin{equation*}
  (a+bj)\cdot(c+dj) = ac + ad\,j + cb\,j - bd
\end{equation*}
Reordenando y agrupando se obtiene el resultado inicial
\begin{equation*}
  (a+bj)\cdot(c+dj) = (ac-bd) + (ad+cb)j
\end{equation*}

\subsection{Propiedades}

Se deduce de las operaciones suma y producto, para todo $z_1,z_2,z_3 \in \mathbb{C}$, que se cumplen las siguientes propiedades
\begin{enumerate}
  \item Conmutativa respecto de la suma: $z_1 + z_2 = z_2 + z_1$
  \item Asociativa respecto de la suma: $z_1 + (z_2 + z_3) = (z_1 + z_2) + z_3$
  \item Elemento neutro de la suma: $u=(0,0)$ tal que $u + z_1 = z_1 + u = z_1$
  \item Elemento opuesto: para $z_1$ su opuesto $z_1'$ es aquel que cumple $z_1 + z_1' = z_1' + z_1 = u$
  \item Conmutativa respecto del producto: $z_1 \cdot z_2 = z_2 \cdot z_1$
  \item Asociativa respecto del producto: $z_1 \cdot (z_2 \cdot z_3) = (z_1 \cdot z_2)\cdot z_3$
  \item Elemento neutro del producto: $p=(1,0)$ tal que $p\cdot z_1 = z_1 \cdot p = z_1$
\end{enumerate}

\subsection{La unidad imaginaria}

Se denomina unidad imaginaria $j$ al número complejo $(0,1)$ que satisface la ecuación
$$
j = \sqrt{-1}
$$
o lo que es lo mismo
$$
j^2 = -1
$$
pues, usando la definición de multiplicación para números complejos
$$
w=(0,1)\cdot(0,1)=((0-1),(0+0))
$$
Resultando que $\Re(w)=-1$ y $\Im(w)=0$. De aquí se pueden obtener los siguientes resultados\footnote{$\Re$ representa la parte real de $w$ e $\Im$ su parte imaginaria}
\begin{itemize}
  \item $j^0 = 1$
  \item $j^1 = j$
  \item $j^2 = -1$
  \item $j^3 = j^2\cdot j= -j$
\end{itemize}
Y, para potencias mayores, se repiten los resultados
\begin{itemize}
  \item $j^4=j^3\cdot j = -j\cdot j = -1 \cdot j^2 = 1$
  \item $j^5=j^4\cdot j = 1 \cdot j = j$
  \item $j^6=j^5\cdot j = j \cdot j = j^2 = -1$
  \item $j^7=j^6\cdot j = -1 \cdot j = -j$
\end{itemize}
Y así sucesivamente.

\subsection{Potencia natural de un numero complejo}

Hasta ahora hemos utilizado la potencia en números complejos, sin embargo no la hemos definido. Se define de la misma forma que para números reales. Dado un número $z\in\mathbb{C}$ y otro número $n\in\mathbb{N}$, su potencia $n$-ésima es
$$
z^n = z \cdot z^{n-1}
$$
Y las definiciones de partida son
\begin{align*}
  &z^0 = 1 \text{ con } z\neq 0 \quad \text{y}\\ &z^1 = z
\end{align*}
Veamos un ejemplo.
\begin{example}
  Dado $z=4+5j$ se pide calcular $z^2$.

  Para calcular $z^2$ aplicamos la definición:
  \begin{align*}
    z^2 &= z \cdot z^1 = z \cdot z \\ 
    (4+5j)^2 &= (4+5j)\cdot (4+5j)^1 = (4+5j)\cdot(4+5j)
  \end{align*}
  Aquí podemos aplicar la definición de producto y obtenemos
  \begin{align*}
    (4+5j)\cdot(4+5j)=(16-25)+(20+20)j=-9+40j
  \end{align*}
  O lo que es lo mismo, aplicar el binomio de Newton
  $$
  (4+5j)^2 = 16 + 2\cdot4\cdot5j + 25j^2 = 16-25 + 40j = -9 + 40j
  $$
  Para otros exponentes como $z^3$ o $z^4$ se puede utilizar el binomio de Newton y agrupar términos para obtener la expresión resultante.
\end{example}

Si nos proporcionan un número complejo en forma exponencial, entonces, la aplicación es la misma. Veamos un ejemplo
\begin{example}
  Dado $z=2e^{j\pi/6}$ se desea obtener $z^2$. Procedemos de la misma forma que en la forma binómica. Partiendo de la definición
  \begin{align*}
    (2e^{j\pi/6})^2 = 2^2 (e^{j\pi/6})^2 = 4 e^{2j\pi/6} = 4 e^{j\pi/3}
  \end{align*}
  Aquí el resultado es mucho más directo. Podemos generalizar a cualquier exponente de forma sencilla
  \begin{align*}
    (re^{j\theta})^n = r^n ~ e^{jn\theta}
  \end{align*}
\end{example}
Por lo que, es conveniente utilizar un complejo en su forma exponencial para resolver potencias.

\subsection{Potencias no naturales}

Ya vimos que la idea de ``potencia'' es una forma compacta de escribir multiplicaciones repetidas. Pero, como suele ocurrir en matemáticas, lo que empieza con un caso sencillo se extiende a casos más generales. 

Dado un número $a\in\mathbb{C}$, se define la potencia para

\subsubsection{Exponentes enteros negativos:}
Para \(n \in \mathbb{N}\),
\[
a^{-n} = \frac{1}{a^n}, \quad a \neq 0.
\]

\subsubsection{Exponentes racionales:}
Si \(n \in \mathbb{N}\) y \(m \in \mathbb{Z}\), definimos
\[
a^{\tfrac{m}{n}} = \sqrt[n]{a^m}, \quad \text{con } a \geq 0 \text{ si } n \text{ es par}.
\]
Así, por ejemplo, \(9^{1/2} = \sqrt{9} = 3\).

\subsubsection{Exponentes reales y complejos:}

Aquí entra un matiz profundo. Para \(w,z \in \mathbb{C}\), definimos:
\[
w^z = e^{z  \log(w)},
\]
donde \(\log(w)\) es el \textbf{logaritmo complejo} de \(w\). Y, como veremos más adelante, el logaritmo complejo no es único, porque:
\[
\log(w) = \ln|w| + j(\arg(w) + 2k\pi), \quad k \in \mathbb{Z}.
\]
Eso significa que, salvo que restrinjamos la rama del logaritmo, la potencia de un número complejo en general no está bien definida de manera única: es multivaluada. Veamos un ejemplo.

\begin{example}
  Un típico ejemplo, sencillo suele ser $j^j$. Aquí tenemos que
  \[
  j^j = e^{j \log(j)}.
  \]
  Como \(\log(j) = j\left(\tfrac{\pi}{2} + 2k\pi\right)\), obtenemos:
  \[
  j^j = e^{-\tfrac{\pi}{2} - 2k\pi}, \quad k \in \mathbb{Z}.
  \]
  De ahí surge algo muy loco: \(j^j\) no solo es real, sino que tiene infinitos valores reales positivos (la principal es \(e^{-\pi/2}\)).
\end{example}

\subsection{Representación gráfica}

Por la definición del conjunto $\mathbb{C}$, como el conjunto de todos los pares ordenados de números reales, se puede representar gráficamente a un número complejo $z=(a,b)$ en un sistema de ejes cartesianos.
\begin{figure}[ht]
  \centering
  \begin{tikzpicture}[>=stealth,scale=2]

    % Ejes
    \draw[->] (-0.2,0) -- (2,0) node[below right] {$\Re(z)$};
    \draw[->] (0,-0.2) -- (0,2) node[above left] {$\Im(z)$};

    % Coordenadas del punto
    \def\a{1.5}
    \def\b{1.2}

    % Punto z
    \filldraw[blue] (\a,\b) circle (1pt) node[above right] {$z=(a,b)$};

    % Proyecciones sobre los ejes
    \draw[dashed] (\a,0) -- (\a,\b) -- (0,\b);

    % Vectores
    \draw[->,thick,blue] (0,0) -- (\a,\b) node[midway,above left] {$r$};

    % Etiquetas a y b
    \node[below] at (\a,0) {$a$};
    \node[left]  at (0,\b) {$b$};

    % Ángulo theta
    \draw[->] (0.8,0) arc (0:atan(\b/\a):0.8);
    \node at (0.6,0.35) {$\theta$};

  \end{tikzpicture}
  \caption{Representación gráfica de un número complejo}
  \label{fig:grafica_de_un_complejo}
\end{figure}

Cada punto del plano representa un número complejo $z$. En la figura \ref{fig:grafica_de_un_complejo} se ha representado $\Re(z)=a$, $\Im(z)=b$, el ángulo con respecto al eje positivo de la parte real $\theta$ y $r=\sqrt{a^2+b^2}$. 

Al representar un número complejo como $z=(r,\theta)$ se dice que este está expresado en forma polar. Esto significa que precisamos el módulo del número complejo $r$ (o tamaño) y su ángulo con respecto al eje positivo de abscisas. 

\subsection{Complejo conjugado}

Dado un complejo $z=(a,b)$, $\bar{z}$ es su conjugado si se cumple que $\bar{z}=(a,-b)$. Es decir la parte imaginaria posee signo opuesto.

El uso del conjugado de un número está en sus propiedades
\begin{itemize}
  \item El producto de un número complejo y su conjugado es un número real.
  \item La suma de dos números complejos conjugados es igual al duplo de la parte real.
\end{itemize}

\subsection{Cociente entre números complejos}

Sean $z_1=a+bj$ y $z_2=c+dj$ números complejos, el cociente entre ellos se obtiene tras multiplicar y dividir por el conjugado del denominador. Es decir
$$
\frac{z_1}{z_2} = \frac{z_1}{z_2}\cdot \frac{\bar{z}_2}{\bar{z}_2}
$$
Como $\bar{z}_2/\bar{z}_2=1+0j$ (que es el elemento neutro multiplicativo) no estamos modificando el producto. Sin embargo podemos aplicar las propiedades de los complejos conjugados
\begin{align*}
  \frac{z_1\cdot \bar{z}_2}{z_2 \cdot \bar{z}_2}
\end{align*}
Al hacer esto, si observamos, el denominador resulta ser un número real (por propiedades de complejos conjugados). Veamos un ejemplo.
\begin{example}
  Realizar el cociente entre $z_1=2+4j$ y $z_2=-1+3j$. 

  Entonces, planteamos el cociente:
  $$
  \frac{2+4j}{-1+3j} = \frac{2+4j}{-1+3j}\cdot\frac{-1-3j}{-1-3j}
  $$
  Ahora aplicamos propiedades de complejos conjugados
  $$
  \frac{(2+4j)(-1-3j)}{(-1+3j)(-1-3j)}=\frac{(-2+12)+(-6-4)j}{(1+9)+(3-3)j} = \frac{1}{10}(10-10j) = 1-j
  $$
\end{example}

Realizar el cociente usando alguna expresión polar es, al igual que el producto, más directo. Veamos un ejemplo usando la forma exponencial.
\begin{example}
  Realizar el cociente entre $z_1=2e^{j\pi/3}$ y $z_2=3e^{j2\pi/3}$. Entonces, escribimos el cociente
  $$
  \frac{z_1}{z_2}=\frac{2e^{j\pi/3}}{3e^{j2\pi/3}}
  $$
  Sin embargo, en este caso vemos que no hay necesidad de multiplicar y dividir por el complejo conjugado. Podemos utilizar directamente propiedades de potencias.
  $$
  \frac{2}{3}\cdot e^{j\pi/3 - j2\pi/3} = \frac{2}{3}\cdot e^{-j\pi/3}
  $$
\end{example}

\subsection{Radicación}

Cuando vimos los exponentes racionales, estos pueden resolverse aplicando radicación. Sin embargo, si no definimos cómo se resuelve la radicación para números complejos, entonces no podríamos encontrar la solución.


