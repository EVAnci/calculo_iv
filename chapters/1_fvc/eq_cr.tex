\section{Ecuaciones de Cauchy-Riemanm}

Las ecuaciones de Cauchy-Riemann (CR) nos permiten saber si una función es analítica. En términos generales, una función $f$ es analítica en un dominio $D$ si y solo si cumplen
\begin{equation}
  \frac{\partial u}{\partial x} = \frac{\partial v}{\partial y} \qquad \frac{\partial u}{\partial y} = -\frac{\partial v}{\partial x}
  \label{eq:cauchy-riemann}
\end{equation}
en todos los puntos de $D$. Las ecuaciones \ref{eq:cauchy-riemann} se denominan las ecuaciones de Cauchy-Riemann.\footnote{Usualmente se usa la notación $u_x=\frac{\partial u}{\partial x}$ para indicar la derivada parcial respecto a $x$ en este caso. Lo mismo se usa para la variable $y$ y para la función $v$.}

\begin{example}
  $f(z)=z^2=x^2-y^2+2jxy$ es analítica para todo $z$ y $u=x^2-y^2$ y $v=2xy$. Verificamos que cumple con las ecuaciones \ref{eq:cauchy-riemann}:
  \begin{gather*}
    u_x = 2x \quad u_y = -2y \qquad v_x = 2y \quad v_y = 2x
  \end{gather*}
  Vemos que
  \begin{align*}
    u_x = v_y \qquad u_y = -v_x
  \end{align*}
  Por tanto, cumple las ecuaciones de Cauchy-Riemann.
\end{example}

\subsection[Demostración de las ecuaciones de Cauchy-Riemann]{Demostración de las ecuaciones de\\Cauchy-Riemann}

\begin{theorem}[Ecuaciones de Cauchy-Riemann]\label{teo:ecuaciones_de_cr}
Sea $f(z) = u(x, y) + jv(x, y)$ una función definida y continua en una vecindad de un punto $z_0 = x_0 + jy_0$, y diferenciable en $z_0$.  
Entonces existen las derivadas parciales de primer orden de $u$ y $v$ en $(x_0, y_0)$, y satisfacen las ecuaciones de Cauchy-Riemann \ref{eq:cauchy-riemann}.
\begin{equation*}
  u_x=v_y \qquad u_y=-v_x
\end{equation*}

Además, si $f$ es analítica en un dominio $D$, entonces estas igualdades se verifican en todos los puntos de $D$.
\end{theorem}

\textit{Demostración}: Por hipótesis, existe la derivada de $f(z)$ en $z_0$. Está dada por
\begin{equation}\label{eq:demostracion_de_cr}
  f'(z_0)=\lim_{\Delta z \to 0} \frac{f(z_0+\Delta z)-f(z_0)}{\Delta z}
\end{equation}
La idea de la demostración es muy simple. Por la definición del límite (sección \ref{sec:limite}), es posible hacer que $\Delta z$ tienda a cero a lo largo de cualquier trayectoria en una vecindad de $z_0$. Por tanto, es posible elegir las dos trayectorias \texttt{I} y \texttt{II} en la figura \ref{fig:trayectorias_de_demostracion_de_cr} e igualar los resultados.
\begin{figure}[ht]
  \centering
  \begin{tikzpicture}[scale=0.7]
    \draw[->] (-1,0)--(5,0) node[below right] {$x$};
    \draw[->] (0,-1)--(0,5) node[above left] {$y$};

    \draw[dashed] (1,1) rectangle (4,4);
    
    \draw[gray, dashed] (1,1) -- (1,-0.2);
    \node[below] at (1,0) {\scriptsize$\color{gray}x_0$};

    \draw[gray, dashed] (1,1) -- (-0.2,1);
    \node[left] at (0,1) {\scriptsize$\color{gray}y_0$};

    \filldraw[red] (1,1) circle (2pt) node[below left] {$z_0$};
    \filldraw[blue] (4,4) circle (2pt) node[above right] {$z_0+\Delta z$};
    
    \node[below right] at (3,1) {\scriptsize\texttt{I}};
    \node[above left] at (2,4) {\scriptsize\texttt{II}};
  \end{tikzpicture}
  \caption{Trayectorias en la ecuación \ref{eq:demostracion_de_cr}}
  \label{fig:trayectorias_de_demostracion_de_cr}
\end{figure}
Al comparar las partes reales se obtiene la primera ecuación de CR, y al comparar las partes imaginarias se obtiene la segunda ecuación.

Se escribe $\Delta z = \Delta x + j \Delta y$. En términos de $u$ y $v$, la derivada $f'(z)$ de la ecuación \ref{eq:demostracion_de_cr} se vuelve
\begin{equation*}
  \lim_{\Delta z \to 0} 
    \frac
      {\left[ 
        u(x_0+\Delta x, y_0+\Delta y) + jv(x_0+\Delta x, y_0 +\Delta y) 
      \right] - \left[
        u(x_0,y_0) + jv(x_0,y_0)
      \right]}
      {\Delta x + j\Delta y}
\end{equation*}

Primero vamos a elegir la trayectoria \texttt{I} (figura \ref{fig:trayectorias_de_demostracion_de_cr}). Como $\Delta z \to 0$, entonces partimos del punto $z_0+\Delta z$. Por tanto, primero se deja $\Delta y \to 0$, y luego $\Delta x \to 0$ resultando:
\begin{align*}
  \lim_{\Delta x \to 0}\frac{[u(x_0+\Delta x,y_0)+jv(x_0+\Delta x, y_0)]-[u(x_0,y_0)+jv(x_0,y_0)]}{\Delta x}
\end{align*}
Separando $u$ y $v$, queda
\begin{equation*}
  f'(z_0)=\lim_{\Delta x \to 0} \frac{u(x_0+\Delta x,y_0)-u(x_0,y_0)}{\Delta x} + j\frac{v(x_0+\Delta x,y_0)-v(x_0,y_0)}{\Delta x}
\end{equation*}
Y como hemos dicho que $f$ es una función derivable en $z_0$:
\begin{equation}\label{eq:demostracion_cr_parcial_de_x}
  \boxed{f'(z_0)=[u_x]_{x_0} + j[v_x]_{x_0}}
\end{equation}
De manera semejante, si se elige la trayectoria \texttt{II}, primero se deja que $\Delta x \to 0$ y luego que $\Delta y \to 0$. Una vez que $x$ es cero, se tiene que $\Delta z = j\Delta y$, de modo que se obtiene
\begin{equation*}
  f'(z_0) = \lim_{\Delta y \to 0}\frac{u(x_0,y_0+\Delta y)-u(x_0,y_0)}{j\Delta y} + j\frac{v(x_0,y_0+\Delta y)-v(x_0,y_0)}{j\Delta y}
\end{equation*}
Debido a que $f'(z_0)$ existe, entonces los dos límites de la derecha existen y proporcionan las derivadas parciales de $u$ y $v$ con respecto a $y$; al observar que $1/j=-j$, entonces se obtiene
\begin{equation}\label{eq:demostracion_cr_parcial_de_y}
  \boxed{f'(z_0) = -j[u_y]_{y_0} + [v_y]_{y_0}}
\end{equation}
Así, la existencia de la derivada $f'(z)$ implica la existencia de las cuatro derivadas parciales en \ref{eq:demostracion_cr_parcial_de_x} y \ref{eq:demostracion_cr_parcial_de_y}. Al igualar las partes de $u_x$ y $v_y$ en \ref{eq:demostracion_cr_parcial_de_x} y \ref{eq:demostracion_cr_parcial_de_y} es posible obtener la primera ecuación de CR. Al igualar las partes imaginarias, se obtiene la segunda ecuación. Lo anterior demuestra la primera afirmación del teorema \ref{teo:ecuaciones_de_cr} e implica la segunda debido a la definición de analiticidad.

\begin{example}
  Para $f(z)=\bar{z}=x-jy$ se tiene $u=x$, $v=-y$ y se observa que
  \begin{gather*}
    u_x=1 \quad v_y = -1 \\ 
    u_x \neq v_y
  \end{gather*}
  Observar el ahorro en esfuerzo de cómputo.
\end{example}

\begin{theorem}[Ecuaciones de Cauchy-Riemann]
  Si dos funciones continuas con valores reales $u(x,y)$ y $v(x,y)$ de dos variables reales $x$ e $y$ tienen primeras derivadas parciales continuas que satisfacen las ecuaciones de Cauchy-Riemann en algún dominio $D$, entonces la función compleja $f(z)=u(x,y)+jv(x,y)$ es analítica en $D$.
  \label{teo:eq_cr_necesaria_suficiente}
\end{theorem}

Las ecuaciones de Cauchy-Riemann constituyen la condición necesaria para que una función sea analítica en un punto. Diremos que es necesaria pero no suficiente, aunque algunos textos las consideran condición necesaria y suficiente por el teorema \ref{teo:eq_cr_necesaria_suficiente}. Veremos a continuación que la restricción de analiticidad se puede perfeccionar. 

Si una función $f(z)$ es analítica en un dominio $D$ entonces $u$ y $v$ satisfacen la ecuación de \textit{Laplace}. Así, diremos que satisfacer la ecuación de Laplace es suficiente para ser analítica, y satisfacer las ecuaciones de Cauchy-Riemann es necesario. He aquí, la condición \textit{necesaria y suficiente}. 

De momento no se preocupe por la ecuación de Laplace. Su enunciado y demostración se verá en la sección \ref{sec:laplace_eq}

Antes de ver la condición suficiente, veamos un ejemplo más de otra clase de problemas que pueden resolverse con las ecuaciones de CR.
\begin{example}
  Encontrar la función analítica más general $f(z)$ cuya parte real sea $u=x^2-y^2-x$.

  \textit{Solución}: Por la primera ecuación de CR,
  \[
    v_y = u_x = 2x-1
  \]
  Entonces, podemos integrar respecto de $y$:
  \begin{align*}
    \int v_y \, dy &= \int 2x -1 \, dy \\ 
    v(x,y) + c_1 &= (2x-1)(y+g(x)) \\ 
    v(x,y) + c_1 &= 2xy - y + 2xg(x) - g(x) 
  \end{align*}
  y si escribimos $k_1=-c_1$, resulta
  \begin{equation}\label{eq:ejemplo_cr_eq_1}
    v(x,y) = 2xy - y + 2xg(x) - g(x) + k_1
  \end{equation}
  donde $g(x)$ es una posible función de $x$ que se ha perdido al derivar $v$ respecto de $y$. Por otra parte, de la segunda ecuación de CR, obtenemos
  \begin{align*}
    v_x &= -u_y = 2y \\ 
    \int v_x \,dx &= \int 2y \, dx \\ 
    v(x,y) + c_2 &= 2y (x + h(y)) 
  \end{align*}
  y si escribimos $k_2=-c_2$, resulta
  \begin{equation}\label{eq:ejemplo_cr_eq_2}
    v(x,y) = 2yx + 2yh(y) + k_2
  \end{equation}
  Aquí vemos que, comparando las ecuaciones \ref{eq:ejemplo_cr_eq_1} y \ref{eq:ejemplo_cr_eq_2}, tenemos que $g(x)=0$ y $h(y)=-1/2$. Por otro lado vemos que $k_1 = k_2 = k$, es una constante arbitraria. Entonces $v(x,y)=2xy - y + k$. Por lo tanto resulta:
  \begin{align*}
    f(x,y) &= u(x,y) + jv(x,y) \\ 
    f(x,y) &= (x^2-y^2-x) + j(2xy - y + k) \\ 
    f(x,y) &= (x^2 + j2xy - y^2) - (x+jy) + jk \\ 
    f(x,y) &= (x+jy)^2 - (x+jy) + jk
  \end{align*}
  Entonces, reemplazando $z=(x+jy)$ en la ecuación, resulta
  \[
    \boxed{f(z) = z^2 - z + jk}
  \]
  donde $k$ es una constante arbitraria.
\end{example}

\subsection{Forma polar de las ecuaciones de CR}

Las ecuaciones de Cauchy-Riemann también pueden ser expresadas en coordenadas polares. En general trabajaremos con la forma binómica, sin embargo se deja la expresión de las ecuaciones en coordenadas polares ya que puede ser útil para algunos casos.
\begin{equation}
  u_r = \frac{1}{r}v_\theta, \qquad v_r =  -\frac{1}{r}u_\theta
\end{equation} 
con $r>0$.

\subsection{Ecuación de Laplace. Funciones armónicas}\label{sec:laplace_eq}

Una de las razones principales de la gran importancia práctica del análisis complejo en las matemáticas aplicadas a la ingeniería resulta del hecho que tanto la parte real como la parte imaginaria de una función analítica satisfacen la ecuación diferencial más importante en física, la ecuación de Laplace, que aparece en la teoría de la gravitación, electrostática, dinámica de fluidos, conducción del calor, etc.

\begin{theorem}[Ecuación de Laplace]
  Si $f(z)=u(x,y)+jv(x,y)$ es analítica en un dominio $D$, entonces $u$ y $v$ satisfacen la ecuación de Laplace en $D$ y tienen segundas derivadas parciales continuas en $D$. Es decir, se cumple que
  \begin{equation}\label{eq:laplace_u}
    \nabla^2 u = u_{xx} + u_{yy} = 0
  \end{equation}
  (aquí $\nabla^2$ es el \href{https://en.wikipedia.org/wiki/Laplace_operator}{operador laplaciano}) y,
  \begin{equation}\label{eq:laplace_v}
    \nabla^2 = v_{xx} + v_{yy} = 0
  \end{equation}
\end{theorem}

\begin{proof}
  Al derivar $u_x = v_y$ con respecto a $x$ y $u_y=-v_x$ con respecto a $y$, se tiene
  \begin{equation}\label{eq:cr_laplace_proof}
    u_{xx}=v_{xy}, \qquad u_{yy} = -v_{xy}
  \end{equation}
  Así, la derivada de una función analítica es en sí misma analítica, como se demostrará después. Lo anterior implica que $u$ y $v$ tienen derivadas parciales continuas de todos los órdenes; en particular, las segundas derivadas parciales mezcladas son iguales; $v_{yx}= v_{xy}$. Al sumar \ref{eq:cr_laplace_proof} se obtiene entonces \ref{eq:laplace_u}.
  \begin{gather*}
    u_{xx}-v_{xy}=0, \qquad u_{yy}+v_{xy}=0 \\ 
    u_{xx}+u_{yy}+v_{xy}-v_{xy}=0 \\ 
    u_{xx}+u_{yy}=\nabla^2 u=0 
  \end{gather*}
  De manera semejante, \ref{eq:laplace_v} se obtiene al derivar $u_x=v_y$ con respecto a $y$ y $u_y=-v_x$ con respecto a $x$ y restando, usando $u_{xy}=u_{yx}$.
\end{proof}

Las soluciones de la ecuación de Laplace que tienen derivadas parciales de segundo orden continuas se denominan \textbf{funciones armónicas} y su teoría se denomina \textit{teoría del potencial}. Por tanto, las partes real e imaginaria de una función analítica son funciones armónicas.

Si dos funciones armónicas $u$ y $v$ satisfacen las ecuaciones de Cauchy-Riemann en un dominio $D$, entonces son las partes real e imaginaria de una función analítica $f$ en $D$. Entonces, se dice que $v$ es la \textbf{función armónica conjugada} de $u$ en $D$.\footnote{Aquí, el uso de la palabra ``conjugada'' es diferente al empleado para definir $\bar{z}$.} Veamos esto con un ejemplo.

\begin{example}[Función armónica conjugada]
  Comprobar que $u=x^2-y^2 -y$ es armónica en todo el plano complejo y encontrar una función armónica conjugada $v$ de $u$.

  \textit{Solución}: $\nabla^2 u =0$ por cálculo directo. Luego, $u_x=2x$, y por otro lado, $u_y=-2y-1$. Por tanto, una conjugada $v$ de $u$ debe satisfacer
  \begin{equation*}
    v_y = u_x = 2x,\qquad v_x=-u_y = 2y+1
  \end{equation*}
  Al integrar la primera ecuación con respecto a $y$, y derivando el resultado con respecto a $x$, se obtiene
  \begin{equation*}
    v=2xy + h(x), \qquad v_x = 2y + \frac{dh}{dx}
  \end{equation*}
  Al comparar el resultado anterior con la segunda ecuación se observa que $dh/dx=1$. Con lo anterior se obtiene $h(x)=x+c$. Así $v=2xy+x+c$ es la armónica conjugada más general de una $u$ dada. La función analítica correspondiente es
  \[
    f(z)=u+jv=x^2-y^2-y+j(2xy+x+c)=z^2 +jz +jc
  \]
\end{example}

Entonces, a modo de resumen decimos que $u$ y $v$
\begin{itemize}
  \item son armónicas si verifican la ecuación de Laplace y tienen derivadas segundas continuas,
  \item $v$ es la armónica conjugada de $u$ si $u$ y $v$ están dispuestas como parte real e imaginaria de una función de variable compleja analítica (y por ende verifican las ecuaciones de Cauchy-Riemann).
\end{itemize}

Un detalle importante es que si tenemos una función $f(x,y)=u(x,y)+jv(x,y)$, entonces $u$ no puede ser armónica conjugada de $v$. Solo $v$ puede ser armónica conjugada de $u$. La definición implica que la conjugada armónica es específicamente la parte imaginaria ($v$) cuando la parte real ($u$) es la función dada. 

Si se nos proporciona la parte compleja de $f$ ($v$), y se nos pide encontrar la conjugada o la función que hace a $v$ la conjugada de otra función incógnita $u$, entonces se procede de la misma forma, pero en vez de buscar la conjugada $v$, buscamos la función que hace a $v$ la conjugada de otra función $u$.
