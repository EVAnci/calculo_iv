\section{Límite, derivada y función analítica}

El análisis complejo estudia funciones complejas que son diferenciables en algún dominio. Por lo tanto, primero es necesario establecer qué se entiende por función compleja y luego definir los conceptos de límite y derivada en los complejos.

\subsection{Función Compleja}

Recordando las funciones reales, una función real $f$ definida sobre un conjunto $S$ de números reales es una regla de correspondencia que asigna a todo $x$ en $S$ un número real $f(x)$, denominado \textit{valor} de $f$ en $x$.

Ahora, en los complejos, $S$ es un conjunto de números \textit{complejos}, y una \textbf{función} $f$ definida sobre $S$ es una regla que asigna a cada $z$ en $S$ un número complejo $w$, denominado \textit{valor} de $f$ en $z$. Se escribe
\begin{equation*}
  w=f(z)
\end{equation*}
Aquí, $z$ varía en $S$ y se denomina \textbf{variable compleja}. El conjunto $S$ se denomina \textbf{dominio} \textit{de definición} de $f$.

\begin{example}
  $w=f(z)=z^2+3z$ es una función compleja definida para todo $z$; es decir, su dominio $S$ es todo el plano complejo.
\end{example}

El conjunto de todos los valores de una función $f$ se denomina rango de $f$. $w$ es complejo y se escribe $w=u+jv$, donde $u$ y $v$ son las partes real e imaginaria, respectivamente. Así, $w$ depende de $z=x+jy$. Por tanto, $u$ se vuelve una función real de $x$ y de $y$, así como lo hace $v$. Por tanto, es posible escribir
\begin{equation*}
  \boxed{w=f(z)=u(x,y)+jv(x,y)}
\end{equation*}
Lo anterior muestra que una función compleja $f(z)$ es equivalente a un par de funciones reales $u(x,y)$ y $v(x,y)$, cada una de las cuales depende de las otras dos variables reales $x$ e $y$.
