\section{Funciones de variable compleja}

Las funciones de variable compleja (FVC), se caracterizan por tener argumento complejo en lugar de argumento real.

Por esta razón, la propia función se constituye también en una magnitud compleja, ya que, al ser así, dispondrá de parte real y parte imaginaria.

Se definen en general como
$$
w=w(z)
$$
donde la variable independiente $z$ es de naturaleza compleja, vale decir que es $z=x+jy$ donde $x$ e $y$ son valores reales que constituyen las partes real e imaginaria de la variable respectivamente.
