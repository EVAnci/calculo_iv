\section{Límite, derivada y función analítica}

El análisis complejo estudia funciones complejas que son diferenciables en algún dominio. Por lo tanto, primero es necesario establecer qué se entiende por función compleja y luego definir los conceptos de límite y derivada en los complejos.

\subsection{Función Compleja}

Recordando las funciones reales, una función real $f$ definida sobre un conjunto $S$ de números reales es una regla de correspondencia que asigna a todo $x$ en $S$ un número real $f(x)$, denominado \textit{valor} de $f$ en $x$.

Ahora, en los complejos, $S$ es un conjunto de números \textit{complejos}, y una \textbf{función} $f$ definida sobre $S$ es una regla que asigna a cada $z$ en $S$ un número complejo $w$, denominado \textit{valor} de $f$ en $z$. Se escribe
\begin{equation*}
  w=f(z)
\end{equation*}
Aquí, $z$ varía en $S$ y se denomina \textbf{variable compleja}. El conjunto $S$ se denomina \textbf{dominio} \textit{de definición} de $f$.

\begin{example}
  $w=f(z)=z^2+3z$ es una función compleja definida para todo $z$; es decir, su dominio $S$ es todo el plano complejo.
\end{example}

El conjunto de todos los valores de una función $f$ se denomina rango de $f$. $w$ es complejo y se escribe $w=u+jv$, donde $u$ y $v$ son las partes real e imaginaria, respectivamente. Así, $w$ depende de $z=x+jy$. Por tanto, $u$ se vuelve una función real de $x$ y de $y$, así como lo hace $v$. Por tanto, es posible escribir
\begin{equation*}
  \boxed{w=f(z)=u(x,y)+jv(x,y)}
\end{equation*}
Lo anterior muestra que una función compleja $f(z)$ es equivalente a un par de funciones reales $u(x,y)$ y $v(x,y)$, cada una de las cuales depende de las otras dos variables reales $x$ e $y$.
\begin{example}
  Sea $w(z)=z^2 + 3z$, encontrar $u$ y $v$ y calcular el valor de $f$ en $z=1+3j$

  \textit{Solución}:
  \begin{align*}
    z^2 &= (x+jy)^2 = x^2 - y^2 + j\,2xy \\ 
    z^2 &= (x^2 - y^2) + j\, (2xy)
  \end{align*}
  Y por otro lado
  \begin{align*}
    3z = 3(x+jy)
  \end{align*}
  Entonces se tiene que
  \begin{align*}
    \operatorname{Re}(z) =& u(x,y) = x^2 - y^2 + 3x \\ 
    \operatorname{Im}(z) =& v(x,y) = 2xy + 3y
  \end{align*}
  Luego, $f(x,y)= (x^2-y^2+3x) + j(2xy+3y)$. Para $z=1+3j$ se reemplaza:
  \begin{align*}
    f(1,3)=(1^2-3^2+3) + j(2\cdot 3 + 3\cdot 3) \\ 
    f(1,3)=-5+15j
  \end{align*}
  Lo anterior muestra que: $u(1,3)=-5$ y $v(1,3)=15$
\end{example}

\subsection{Límite y continuidad}
\label{sec:limite}

Se dice que $L$ es el \textbf{límite} de una función $f(z)$ cuando $z$ tiende al punto $z_0$, lo que se escribe
\begin{equation}
  \lim_{z\to z_0}f(z)=L
\end{equation}
si $f$ está definida en una vecindad de $z_0$ (excepto quizá en $z_0$ mismo) y si los valores de $f$ están ``próximos'' a $L$ para todo $z$ ``próximo'' a $z_0$.

\begin{figure}[ht]
  \centering
  \begin{subfigure}{0.48\textwidth}
    \centering
    \begin{tikzpicture}[scale=0.7, remember picture]
      % Ejes
      \draw[->] (0,0) -- (0,5) node[above left] {$\Im(z)$};
      \draw[->] (0,0) -- (5,0) node[below right] {$\Re(z)$};

      % Vecindad en el dominio
      \filldraw[blue] (2.5,2.5) circle (2pt) node[below right] {$z_0$};
      \draw[->] (2.5,2.5)--++(200:2) node[midway, above] {$\delta$};
      \draw[red, dashed] (2.5,2.5) circle (2);
      \filldraw (2.2,3.5) circle (2pt) node[left] {$z$};
      \coordinate (zpoint) at (2.2,3.5);
    \end{tikzpicture}
    \caption{Vecindad de $z_0$ en el dominio.}
  \end{subfigure}
  \hfill
  \begin{subfigure}{0.48\textwidth}
    \centering
    \begin{tikzpicture}[scale=0.7, remember picture]
      % Ejes
      \draw[->] (0,0) -- (0,5) node[above left] {$v$};
      \draw[->] (0,0) -- (6,0) node[below right] {$u$};

      % Vecindad en la imagen
      \filldraw[teal] (2.5,1.5) circle (2pt) node[below right] {$L$};
      \draw[->] (2.5,1.5)--++(190:2.4) node[midway, above] {$\varepsilon$};
      \draw[red, dashed] (2.5,1.5) circle (2.4);
      \filldraw (3.5,1.7) circle (2pt) node[right] {$f(z)$};
      \coordinate (fpoint) at (3.5,1.7);
    \end{tikzpicture}
    \caption{Vecindad de $L=f(z_0)$ en el codominio.}
  \end{subfigure}

  \begin{tikzpicture}[remember picture, overlay]
    \draw[->, dashed, gray!60!black]
      ($(zpoint)$)
      .. controls ($(zpoint)+(3,0.3)$) ..
      ($(fpoint)$)
      node[midway, above, sloped=false] {$f$};
  \end{tikzpicture}
  \caption{Relación entre la vecindad en el dominio y su imagen por $f$.}
  \label{fig:limite_complejo}
\end{figure}
Aquí la definición de límite es parecida a el límite para funciones de varias variables reales; es decir, aquí por definición, $z$ puede tender a $z_0$ \textbf{desde cualquier dirección} (como muestra la figura \ref{fig:limite_complejo}) en el plano complejo. Al igual que para variable real, si existe un límite, entonces es único.

Se dice que una función $f(z)$ es \textbf{continua} en $z=z_0$ si
\begin{itemize}
  \item Existe el límite en $z_0$: $\lim_{z\to z_0} f(z) = L$
  \item Existe $f$ en $z_0$: $\exists f(z_0) $ para $z_0 \in\text{D}(f)$
  \item El límite y $f(z_0)$ son iguales: $\lim_{z\to z_0} f(z) = L = f(z_0)$
\end{itemize}

Se dice que $f(z)$ es continua en un dominio si es continua en cada uno de los puntos de este dominio.

\subsection{Derivada}

La derivada de una función compleja $f$ en un punto $z_0$ se denota por $f'(z_0)$ y se define como
\begin{equation}
  \boxed{f'(z_0)=\lim_{\Delta z \to 0}\frac{f(z_0+\Delta z)-f(z_0)}{\Delta z}}
  \label{eq:derivada}
\end{equation}
en el supuesto de que este límite existe. Así, se dice entonces que $f$ es diferenciable en $z_0$. 

Si $f$ es diferenciable en cada punto de un conjunto, entonces $f$ es diferenciable en ese conjunto.

Recuérsede que, por la definición del límite, $f(z)$ está definida en alguna vecindad de $z_0$ y, por tanto, $z$ puede aproximarse a $z_0$ (como indica la ecuación \ref{eq:derivada}) desde cualquier dirección y con cualquier trayectoria. Por tanto, diferenciabilidad en $z_0$ significa que, a lo largo de cualquier trayectoria por la cual $z$ tienda a $z_0$, el cociente en \ref{eq:derivada} siempre tiende a un cierto valor, y todos esos valores son iguales. 
\begin{example}
  La función $f(z)=z^2$ es diferenciable para todo $z$ y su derivada es $f'(z)=2z$ porque
  \begin{align*}
    f'(z)&=\lim_{\Delta z \to 0} \frac{(z+\Delta z)^2-z^2}{\Delta z} \\
         &=\lim_{\Delta z \to 0} \frac{\cancel{z^2}+2z\Delta z + \Delta z^2 - \cancel{z^2}}{\Delta z} \\ 
         &=\lim_{\Delta z \to 0} \frac{2z\cancel{\Delta z}}{\cancel{\Delta z}} + \frac{\Delta z^{\cancel{2}}}{\cancel{\Delta z}} \\ 
    f'(z)&=2z
  \end{align*}
\end{example}

Las \textbf{reglas de diferenciación} son las mismas que en cálculo real, ya que sus demostraciones son literalmente iguales. Entonces, generalizando las reglas de derivación tenemos:
\begin{itemize}
  \item $(cf)'(z)=cf(z)'$ 
  \item $(f+g)'(z)=f'(z)+g'(z)$
  \item $(fg)'(z)=f'(z)g(z)+f(z)g'(z)$
  \item $\left(\frac{f}{g}\right)'(z)=\frac{f'(z)g(z)-f(z)g'(z)}{g(z)^2}\qquad g\neq 0$
  \item $(f\circ g)'(z) = f'(g(z))g'(z)$
\end{itemize}
suponiendo que $f$ y $g$ son diferenciables en $z$ y, en el último ítem $f$ es diferenciable en $g(z)$.

Para funciones familiares tales como los polinomios, se aplican las reglas usuales para derivadas. Por ejemplo, si $n$ es un entero positivo y $f(z)=z^n$, entonces $f'(z)=nz^{n-1}$. Cuando desarrolle la función compleja $f(z)=\sin(z)$, verá que $f'(z)=\cos(z)$. 

También, si $f(z)$ es diferenciable en $z_0$, entonces es continua en $z_0$.

\begin{example}
  Es importante observar que existen muchas funciones simples que no tienen derivada en ningún punto. Por ejemplo, $f(z)=\bar{z}=x-jy$ es una de estas funciones. De hecho, si se escribe $\Delta z = \Delta x - j\Delta y$, se tiene 
  \begin{align}
    \frac{f(z+\Delta z) - f(z)}{\Delta z} = \frac{\bar{(z+\Delta z)-\bar{z}}}{\Delta z} = \frac{\bar{\Delta z}}{\Delta z} = \frac{\Delta x - j\Delta y}{\Delta x + j\Delta y}
    \label{ej:eq_no_diferenciable}
  \end{align}
  Si $\Delta y = 0$, lo anterior es 1. Contrariamente, si $\Delta x = 0$ entonces es $-1$. Por tanto, \ref{ej:eq_no_diferenciable} tiende a dos valores distintos cuando se toman distintas trayectorias, y, en consecuencia, el límite no existe.
\end{example}

El ejemplo que se acaba de analizar puede ser sorprendente, aunque simplemente ilustra que la diferenciabilidad de una función compleja es más bien un requisito estricto.

\subsection{Funciones analíticas}

Se trata de funciones que son diferenciables en algún dominio, de modo que es posible hacer ``cálculo en los complejos''. Constituyen el tema fundamental del análisis complejo, y su introducción es el objetivo principal de esta sección. 

\subsection{Definición de analiticidad}

Se dice que una función $f(z)$ es \textit{analítica en un dominio D} si $f(z)$ está definida y es diferenciable en todos los puntos de \textit{D}. Se dice que una función $f(z)$ es analítica en un punto $z=z_0$ en $D$ si $f(z)$ es analítica en una vecindad de $z_0$. Es decir, 
\begin{gather*}
  \text{``}f \text{ es analítica en }z_0 \Leftrightarrow f \text{ es diferenciable en una vecindad de }z_0\text{''} \\ 
  \text{``}f \text{ es analítica en }D \Leftrightarrow f \text{ es diferenciable en todos los puntos de }D\text{''}
\end{gather*}

Así, la analiticidad de $f(z)$ en $z_0$ significa que $f(z)$ tiene una derivada en todos los puntos en alguna vecindad de $z_0$ (incluyendo $z_0$ mismo ya que, por definición, $z_0$ es un punto que pertenece a todas sus vecindades). En palabras simples, decimos que ser analítica en $D$ es ser diferenciable (u holomorfa) en $D$. Es decir, vamos a estudiar funciones que sean diferenciables en un entorno.

