\section{Límite, derivada y función analítica}

El análisis complejo estudia funciones complejas que son diferenciables en algún dominio. Por lo tanto, primero es necesario establecer qué se entiende por función compleja y luego definir los conceptos de límite y derivada en los complejos.

\subsection{Función Compleja}

Recordando las funciones reales, una función real $f$ definida sobre un conjunto $S$ de números reales es una regla de correspondencia que asigna a todo $x$ en $S$ un número real $f(x)$, denominado \textit{valor} de $f$ en $x$.

Ahora, en los complejos, $S$ es un conjunto de números \textit{complejos}, y una \textbf{función} $f$ definida sobre $S$ es una regla que asigna a cada $z$ en $S$ un número complejo $w$, denominado \textit{valor} de $f$ en $z$. Se escribe
\begin{equation*}
  w=f(z)
\end{equation*}
Aquí, $z$ varía en $S$ y se denomina \textbf{variable compleja}. El conjunto $S$ se denomina \textbf{dominio} \textit{de definición} de $f$.

\begin{example}
  $w=f(z)=z^2+3z$ es una función compleja definida para todo $z$; es decir, su dominio $S$ es todo el plano complejo.
\end{example}

El conjunto de todos los valores de una función $f$ se denomina rango de $f$. $w$ es complejo y se escribe $w=u+jv$, donde $u$ y $v$ son las partes real e imaginaria, respectivamente. Así, $w$ depende de $z=x+jy$. Por tanto, $u$ se vuelve una función real de $x$ y de $y$, así como lo hace $v$. Por tanto, es posible escribir
\begin{equation*}
  \boxed{w=f(z)=u(x,y)+jv(x,y)}
\end{equation*}
Lo anterior muestra que una función compleja $f(z)$ es equivalente a un par de funciones reales $u(x,y)$ y $v(x,y)$, cada una de las cuales depende de las otras dos variables reales $x$ e $y$.
\begin{example}
  Sea $w(z)=z^2 + 3z$, encontrar $u$ y $v$ y calcular el valor de $f$ en $z=1+3j$

  \textit{Solución}:
  \begin{align*}
    z^2 &= (x+jy)^2 = x^2 - y^2 + j\,2xy \\ 
    z^2 &= (x^2 - y^2) + j\, (2xy)
  \end{align*}
  Y por otro lado
  \begin{align*}
    3z = 3(x+jy)
  \end{align*}
  Entonces se tiene que
  \begin{align*}
    \operatorname{Re}(z) =& u(x,y) = x^2 - y^2 + 3x \\ 
    \operatorname{Im}(z) =& v(x,y) = 2xy + 3y
  \end{align*}
  Luego, $f(x,y)= (x^2-y^2+3x) + j(2xy+3y)$. Para $z=1+3j$ se reemplaza:
  \begin{align*}
    f(1,3)=(1^2-3^2+3) + j(2\cdot 3 + 3\cdot 3) \\ 
    f(1,3)=-5+15j
  \end{align*}
  Lo anterior muestra que: $u(1,3)=-5$ y $v(1,3)=15$
\end{example}

\subsection{Límite y continuidad}
\label{sec:limite}

Se dice que $L$ es el \textbf{límite} de una función $f(z)$ cuando $z$ tiende al punto $z_0$, lo que se escribe
\begin{equation}
  \lim_{z\to z_0}f(z)=L
\end{equation}
si $f$ está definida en una vecindad de $z_0$ (excepto quizá en $z_0$ mismo) y si los valores de $f$ están ``próximos'' a $L$ para todo $z$ ``próximo'' a $z_0$.

\begin{figure}[ht]
  \centering
  \begin{subfigure}{0.48\textwidth}
    \centering
    \begin{tikzpicture}[scale=0.7, remember picture]
      % Ejes
      \draw[->] (0,0) -- (0,5) node[above left] {$\Im(z)$};
      \draw[->] (0,0) -- (5,0) node[below right] {$\Re(z)$};

      % Vecindad en el dominio
      \filldraw[blue] (2.5,2.5) circle (2pt) node[below right] {$z_0$};
      \draw[->] (2.5,2.5)--++(200:2) node[midway, above] {$\delta$};
      \draw[red, dashed] (2.5,2.5) circle (2);
      \filldraw (2.2,3.5) circle (2pt) node[left] {$z$};
      \coordinate (zpoint) at (2.2,3.5);
    \end{tikzpicture}
    \caption{Vecindad de $z_0$ en el dominio.}
  \end{subfigure}
  \hfill
  \begin{subfigure}{0.48\textwidth}
    \centering
    \begin{tikzpicture}[scale=0.7, remember picture]
      % Ejes
      \draw[->] (0,0) -- (0,5) node[above left] {$v$};
      \draw[->] (0,0) -- (6,0) node[below right] {$u$};

      % Vecindad en la imagen
      \filldraw[teal] (2.5,1.5) circle (2pt) node[below right] {$L$};
      \draw[->] (2.5,1.5)--++(190:2.4) node[midway, above] {$\varepsilon$};
      \draw[red, dashed] (2.5,1.5) circle (2.4);
      \filldraw (3.5,1.7) circle (2pt) node[right] {$f(z)$};
      \coordinate (fpoint) at (3.5,1.7);
    \end{tikzpicture}
    \caption{Vecindad de $L=f(z_0)$ en el codominio.}
  \end{subfigure}

  \begin{tikzpicture}[remember picture, overlay]
    \draw[->, dashed, gray!60!black]
      ($(zpoint)$)
      .. controls ($(zpoint)+(3,0.3)$) ..
      ($(fpoint)$)
      node[midway, above, sloped=false] {$f$};
  \end{tikzpicture}
  \caption{Relación entre la vecindad en el dominio y su imagen por $f$.}
  \label{fig:limite_complejo}
\end{figure}
Aquí la definición de límite es parecida a el límite para funciones de varias variables reales; es decir, aquí por definición, $z$ puede tender a $z_0$ \textbf{desde cualquier dirección} (como muestra la figura \ref{fig:limite_complejo}) en el plano complejo. Al igual que para variable real, si existe un límite, entonces es único.

Se dice que una función $f(z)$ es \textbf{continua} en $z=z_0$ si
\begin{itemize}
  \item Existe el límite en $z_0$: $\lim_{z\to z_0} f(z) = L$
  \item Existe $f$ en $z_0$: $\exists f(z_0) $ para $z_0 \in\text{D}(f)$
  \item El límite y $f(z_0)$ son iguales: $\lim_{z\to z_0} f(z) = L = f(z_0)$
\end{itemize}

Se dice que $f(z)$ es continua en un dominio si es continua en cada uno de los puntos de este dominio.

\subsection{Derivada}

La derivada de una función compleja $f$ en un punto $z_0$ se denota por $f'(z_0)$ y se define como
\begin{equation}
  \boxed{f'(z_0)=\lim_{\Delta z \to 0}\frac{f(z_0+\Delta z)-f(z_0)}{\Delta z}}
\end{equation}
en el supuesto de que este límite existe. Así, se dice entonces que $f$ es diferenciable en $z_0$. Si se escribe $\Delta z = z-z_0$, entonces también se tiene que, como $z=z_0+\Delta z$,
\begin{equation}
  f'(z_0) = \lim_{z\to z_0} \frac{f(z)-f(z_0)}{z-z_0}
  \label{eq:derivada}
\end{equation}
Recuérsede que, por la definición del límite, $f(z)$ está definida en alguna vecindad de $z_0$ y, por tanto, $z$ puede aproximarse a $z_0$ (como indica la ecuación \ref{eq:derivada}) desde cualquier dirección y con cualquier trayectoria. Por tanto, diferenciabilidad en $z_0$ significa que, a lo largo de cualquier trayectorial por la cual $z$ tienda a $z_0$, el cociente en \ref{eq:derivada} siempre tiende a un cierto valor, y todos esos valores son iguales. 
\begin{example}
  La función $f(z)=z^2$ es diferenciable para todo $z$ y su derivada es $f'(z)=2z$ porque
  \begin{align*}
    f'(z)&=\lim_{\Delta z \to 0} \frac{(z+\Delta z)^2-z^2}{\Delta z} \\
         &=\lim_{\Delta z \to 0} \frac{\cancel{z^2}+2z\Delta z + \Delta z^2 - \cancel{z^2}}{\Delta z} \\ 
         &=\lim_{\Delta z \to 0} \frac{2z\cancel{\Delta z}}{\cancel{\Delta z}} + \frac{\Delta z^{\cancel{2}}}{\cancel{\Delta z}} \\ 
    f'(z)&=2z
  \end{align*}
  \label{ej:derivada}
\end{example}

Las \textbf{reglas de diferenciación} son las mismas que en cálculo real, ya que sus demostraciones son literalmente iguales. Entonces, generalizando las reglas de derivación tenemos:
\begin{itemize}
  \item $(cf)'=cf'$ 
  \item $(f+g)'=f'+g'$
  \item $(fg)'=f'g+fg'$
  \item $\left(\frac{f}{g}\right)'=\frac{f'g-fg'}{g^2}$
\end{itemize}
y también se cumple la regla de la cadena.

También, si $f(z)$ es diferenciable en $z_0$, entonces es continua en $z_0$.

\begin{example}
  Es importante observar que existen muchas funciones simples que no tienen derivada en ningún punto. Por ejemplo, $f(z)=\bar{z}=x-jy$ es una de estas funciones. De hecho, si se escribe $\Delta z = \Delta x - j\Delta y$, se tiene 
  \begin{align}
    \frac{f(z+\Delta z) - f(z)}{\Delta z} = \frac{\bar{(z+\Delta z)-\bar[z]}}{\Delta z} = frac{\bar{\Delta z}}{\Delta z} = \frac{\Delta x - j\Delta y}{\Delta x + j\Delta y}
    \label{ej:eq_no_diferenciable}
  \end{align}
  Si $\Delta y = 0$, lo anterior es 1. Contrariamente, si $\Delta x = 0$ entonces es $-1$. Por tanto, \ref{ej:eq_no_diferenciable} tiende a dos valores distintos cuando se toman distintas trayectorias, y, en consecuencia, el límite no existe.
\end{example}

El ejemplo que se acaba de analizar puede ser sorprendente, aunque simplemente ilustra que la diferenciabilidad de una función compleja es más bien un requisito estricto.

\subsection{Funciones analíticas}

Se trata de funciones que son diferenciables en algún dominio, de modo que es posible hacer ``cálculo en los complejos''. Constituyen el tema fundamental del análisis complejo, y su introducción es el objetivo principal de esta sección. 

\subsection{Definición de analiticidad}

Se dice que una función $f(z)$ es \textit{analítica en un dominio D} si $f(z)$ está definida y es diferenciable en todos los puntos de \textit{D}. Se dice que una función $f(z)$ es analítica en un punto $z=z_0$ en $D$ si $f(z)$ es analítica en una vecindad de $z_0$. Es decir, 
\begin{gather*}
  \text{``}f \text{ es analítica en }z_0 \Leftrightarrow f \text{ es diferenciable en una vecindad de }z_0\text{''} \\ 
  \text{``}f \text{ es analítica en }D \Leftrightarrow f \text{ es diferenciable en todos los puntos de }D\text{''}
\end{gather*}

Así, la analiticidad de $f(z)$ en $z_0$ significa que $f(z)$ tiene una derivada en todos los puntos en alguna vecindad de $z_0$ (incluyendo $z_0$ mismo ya que, por definición, $z_0$ es un punto que pertenece a todas sus vecindades). En palabras simples, decimos que ser analítica en $D$ es ser diferenciable (u holomorfa) en $D$. Es decir, vamos a estudiar funciones que sean diferenciables en un entorno.

\section{Ecuaciones de Cauchy-Riemanm}

Las ecuaciones de Cauchy-Riemann (CR) nos permiten saber si una función es analítica. En términos generales, una función $f$ es analítica en un dominio $D$ si y solo si cumplen
\begin{equation}
  \frac{\partial u}{\partial x} = \frac{\partial v}{\partial y} \qquad \frac{\partial u}{\partial y} = -\frac{\partial v}{\partial x}
  \label{eq:cauchy-riemann}
\end{equation}
en todos los puntos de $D$. Las ecuaciones \ref{eq:cauchy-riemann} se denominan las ecuaciones de Cauchy-Riemann.\footnote{Usualmente se usa la notación $u_x=\frac{\partial u}{\partial x}$ para indicar la derivada parcial respecto a $x$ en este caso. Lo mismo se usa para la variable $y$ y para la función $v$.}

\begin{example}
  $f(z)=z^2=x^2=y^2+2jxy$ es analítica para todo $z$ y $u=x^2-y^2$ y $v=2xy$. Verificamos que cumple con las ecuaciones \ref{eq:cauchy-riemann}:
  \begin{gather*}
    u_x = 2x \quad u_y = -2y \qquad v_x = 2y \quad v_y = 2x
  \end{gather*}
  Vemos que
  \begin{align*}
    u_x = v_y \qquad u_y = -v_x
  \end{align*}
  Por tanto, cumple las ecuaciones de Cauchy-Riemann.
\end{example}

\subsection[Demostración de las ecuaciones de Cauchy-Riemann]{Demostración de las ecuaciones de\\Cauchy-Riemann}

\begin{theorem}[Ecuaciones de Cauchy-Riemann]\label{teo:ecuaciones_de_cr}
Sea $f(z) = u(x, y) + jv(x, y)$ una función definida y continua en una vecindad de un punto $z_0 = x_0 + jy_0$, y diferenciable en $z_0$.  
Entonces existen las derivadas parciales de primer orden de $u$ y $v$ en $(x_0, y_0)$, y satisfacen las ecuaciones de Cauchy-Riemann \ref{eq:cauchy-riemann}.
\begin{equation*}
  u_x=v_y \qquad u_y=-v_x
\end{equation*}

Además, si $f$ es analítica en un dominio $D$, entonces estas igualdades se verifican en todos los puntos de $D$.
\end{theorem}

\textit{Demostración}: Por hipótesis, existe la derivada de $f(z)$ en $z_0$. Está dada por
\begin{equation}\label{eq:demostracion_de_cr}
  f'(z_0)=\lim_{\Delta z \to 0} \frac{f(z_0+\Delta z)-f(z_0)}{\Delta z}
\end{equation}
La idea de la demostración es muy simple. Por la definición del límite (sección \ref{sec:limite}), es posible hacer que $\Delta z$ tienda a cero a lo largo de cualquier trayectoria en una vecindad de $z_0$. Por tanto, es posible elegir las dos trayectorias \texttt{I} y \texttt{II} en la figura \ref{fig:trayectorias_de_demostracion_de_cr} e igualar los resultados.
\begin{figure}[ht]
  \centering
  \begin{tikzpicture}[scale=0.7]
    \draw[->] (-1,0)--(5,0) node[below right] {$x$};
    \draw[->] (0,-1)--(0,5) node[above left] {$y$};

    \draw[dashed] (1,1) rectangle (4,4);
    
    \draw[gray, dashed] (1,1) -- (1,-0.2);
    \node[below] at (1,0) {\scriptsize$\color{gray}x_0$};

    \draw[gray, dashed] (1,1) -- (-0.2,1);
    \node[left] at (0,1) {\scriptsize$\color{gray}y_0$};

    \filldraw[red] (1,1) circle (2pt) node[below left] {$z_0$};
    \filldraw[blue] (4,4) circle (2pt) node[above right] {$z_0+\Delta z$};
    
    \node[below right] at (3,1) {\scriptsize\texttt{I}};
    \node[above left] at (2,4) {\scriptsize\texttt{II}};
  \end{tikzpicture}
  \caption{Trayectorias en la ecuación \ref{eq:demostracion_de_cr}}
  \label{fig:trayectorias_de_demostracion_de_cr}
\end{figure}
Al comparar las partes reales se obtiene la primera ecuación de CR, y al comparar las partes imaginarias se obtiene la segunda ecuación.

Se escribe $\Delta z = \Delta x + j \Delta y$. En términos de $u$ y $v$, la derivada $f'(z)$ de la ecuación \ref{eq:demostracion_de_cr} se vuelve
\begin{equation*}
  \lim_{\Delta z \to 0} 
    \frac
      {\left[ 
        u(x_0+\Delta x, y_0+\Delta y) + jv(x_0+\Delta x, y_0 +\Delta y) 
      \right] - \left[
        u(x_0,y_0) + jv(x_0,y_0)
      \right]}
      {\Delta x + j\Delta y}
\end{equation*}

Primero vamos a elegir la trayectoria \texttt{I} (figura \ref{fig:trayectorias_de_demostracion_de_cr}). Como $\Delta z \to 0$, entonces partimos del punto $z_0+\Delta z$. Por tanto, primero se deja $\Delta y \to 0$, y luego $\Delta x \to 0$ resultando:
\begin{align*}
  \lim_{\Delta x \to 0}\frac{[u(x_0+\Delta x,y_0)+jv(x_0+\Delta x, y_0)]-[u(x_0,y_0)+jv(x_0,y_0)]}{\Delta x}
\end{align*}
Separando $u$ y $v$, queda
\begin{equation*}
  f'(z_0)=\lim_{\Delta x \to 0} \frac{u(x_0+\Delta x,y_0)-u(x_0,y_0)}{\Delta x} + j\frac{v(x_0+\Delta x,y_0)-v(x_0,y_0)}{\Delta x}
\end{equation*}
Y como hemos dicho que $f$ es una función derivable en $z_0$:
\begin{equation}\label{eq:demostracion_cr_parcial_de_x}
  \boxed{f'(z_0)=[u_x]_{x_0} + j[v_x]_{x_0}}
\end{equation}
De manera semejante, si se elige la trayectoria \texttt{II}, primero se deja que $\Delta x \to 0$ y luego que $\Delta y \to 0$. Una vez que $x$ es cero, se tiene que $\Delta z = j\Delta y$, de modo que se obtiene
\begin{equation*}
  f'(z_0) = \lim_{\Delta y \to 0}\frac{u(x_0,y_0+\Delta y)-u(x_0,y_0)}{j\Delta y} + j\frac{v(x_0,y_0+\Delta y)-v(x_0,y_0)}{j\Delta y}
\end{equation*}
Debido a que $f'(z_0)$ existe, entonces los dos límites de la derecha existen y proporcionan las derivadas parciales de $u$ y $v$ con respecto a $y$; al observar que $1/j=-j$, entonces se obtiene
\begin{equation}\label{eq:demostracion_cr_parcial_de_y}
  \boxed{f'(z_0) = -j[u_y]_{y_0} + [v_y]_{y_0}}
\end{equation}
Así, la existencia de la derivada $f'(z)$ implica la existencia de las cuatro derivadas parciales en \ref{eq:demostracion_cr_parcial_de_x} y \ref{eq:demostracion_cr_parcial_de_y}. Al igualar las partes de $u_x$ y $v_y$ en \ref{eq:demostracion_cr_parcial_de_x} y \ref{eq:demostracion_cr_parcial_de_y} es posible obtener la primera ecuación de CR. Al igualar las partes imaginarias, se obtiene la segunda ecuación. Lo anterior demuestra la primera afirmación del teorema \ref{teo:ecuaciones_de_cr} e implica la segunda debido a la definición de analiticidad.

\begin{example}
  Para $f(z)=\bar{z}=x-jy$ se tiene $u=x$, $v=-y$ y se observa que
  \begin{gather*}
    u_x=1 \quad v_y = -1 \\ 
    u_x \neq v_y
  \end{gather*}
  Observar el ahorro en esfuerzo de cómputo
\end{example}

Las ecuaciones de Cauchy-Riemann son fundamentales debido a que no sólo son necesarias sino también suficientes para que una función sea analítica. Con más precisión, se cumple el siguiente teorema.

\begin{theorem}[Ecuaciones de Cauchy-Riemann]
  Si dos funciones continuas con valores reales $u(x,y)$ y $v(x,y)$ de dos variables reales $x$ e $y$ tienen primeras derivadas parciales continuas que satisfacen las ecuaciones de Cauchy-Riemann en algún dominio $D$, entonces la función compleja $f(z)=u(x,y)+jv(x,y)$ es analítica en $D$.
\end{theorem}

La demostración de este teorema es más complicada que la del teorema \ref{teo:ecuaciones_de_cr} por lo tanto no se verá.

Veamos un ejemplo más de otra clase de problemas que pueden resolverse con las ecuaciones de CR.
\begin{example}
  Encontrar la función analítica más general $f(z)$ cuya parte real sea $u=x^2-y^2-x$.

  \textit{Solución}: Por la primera ecuación de CR,
  \[
    v_y = u_x = 2x-1
  \]
  Entonces, podemos integrar respecto de $y$:
  \begin{align*}
    \int v_y \, dy &= \int 2x -1 \, dy \\ 
    v(x,y) + c_1 &= (2x-1)(y+g(x)) \\ 
    v(x,y) + c_1 &= 2xy - y + 2xg(x) - g(x) 
  \end{align*}
  y si escribimos $k_1=-c_1$, resulta
  \begin{equation}\label{eq:ejemplo_cr_eq_1}
    v(x,y) = 2xy - y + 2xg(x) - g(x) + k_1
  \end{equation}
  donde $g(x)$ es una posible función de $x$ que se ha perdido al derivar $v$ respecto de $y$. Por otra parte, de la segunda ecuación de CR, obtenemos
  \begin{align*}
    v_x &= -u_y = 2y \\ 
    \int v_x \,dx &= \int 2y \, dx \\ 
    v(x,y) + c_2 &= 2y (x + h(y)) 
  \end{align*}
  y si escribimos $k_2=-c_2$, resulta
  \begin{equation}\label{eq:ejemplo_cr_eq_2}
    v(x,y) = 2yx + 2yh(y) + k_2
  \end{equation}
  Aquí vemos que, comparando las ecuaciones \ref{eq:ejemplo_cr_eq_1} y \ref{eq:ejemplo_cr_eq_2}, tenemos que $g(x)=0$ y $h(y)=-1/2$. Por otro lado vemos que $k_1 = k_2 = k$, es una constante arbitraria. Entonces $v(x,y)=2xy - y + k$. Por lo tanto resulta:
  \begin{align*}
    f(x,y) &= u(x,y) + jv(x,y) \\ 
    f(x,y) &= (x^2-y^2-x) + j(2xy - y + k) \\ 
    f(x,y) &= (x^2 + j2xy - y^2) - (x+jy) + jk \\ 
    f(x,y) &= (x+jy)^2 - (x+jy) + jk
  \end{align*}
  Entonces, reemplazando $z=(x+jy)$ en la ecuación, resulta
  \[
    \boxed{f(z) = z^2 - z + jk}
  \]
  donde $k$ es una constante arbitraria.
\end{example}

\subsection{Forma polar de las ecuaciones de CR}

A veces, es conveniente usar coordenadas polares, por tanto, es útil tener una expresión de las ecuaciones de Cauchy-Riemann usando este sistema.

En coordenadas polares, tenemos que $z=re^{j\theta}$, donde $r=\sqrt{x^2+y^2}$ y $\theta=\arctan(\frac{y}{x})$. En este caso, la función $f(z)$ se puede escribir en términos de $r$ y $\theta$, así que denotamos $f(z)=u(r,\theta)+jv(r,\theta)$.

Para encontrar la expresión de las ecuaciones de CR en coordenadas polares, partimos de las ecuaciones originales, para coordenadas cartesianas:
\[
  u_x = v_y \qquad u_y = -v_x
\]
donde $f(z)=u(x,y)+jv(x,y)$ y $z=x+jy$. En base a ello, escribimos
\[
  x=r\cos(\theta),\quad y=r\sin(\theta)
\]
Ahora, consideramos las derivadas parciales de $u$ y de $v$ con respecto a $x$ e $y$, y se expresan en términos de $r$ y $\theta$.

Para $u$ tenemos
\begin{align*}
  \frac{\partial u}{\partial x} = \frac{\partial u}{\partial r}\frac{\partial r}{\partial x} + \frac{\partial u}{\partial \theta}\frac{\partial \theta}{\partial x}
\end{align*}
