\section{Transformada de Fourier}

En matemáticas se llama transformada al mecanismo que convierte un problema de un tipo a otro, este último presumiblemente más fácil de resolver. El modo de hacerlo es resolver primero el problema transformado, para después transformar de regreso y así obtener la solución del problema original. En el caso de la transformada de Fourier (o transformada de Lapace como se verá más adelante) los problemas con valores iniciales son convertidos en problemas algebraicos, un proceso que se puede ilustrar del modo siguiente
\begin{figure}[ht]
  \centering
  \begin{tikzpicture}[node distance=5mm,
    terminal/.style={
      % The shape:
      rectangle,minimum size=6mm,rounded corners=3mm,
      % The rest
      very thick,draw=black!50,
      top color=white,bottom color=black!20,
      font=\ttfamily
    }]
    \node (start) [terminal] {problema con valores iniciales};
    \node (algebraic) [terminal,below=of start] {problema algebraico};
    \node (solution_alg) [terminal,below=of algebraic] {solución del problema algebraico};
    \node (end) [terminal,below=of solution_alg] {solución del problema con valores iniciales};

    \path (start) edge[->] (algebraic) 
    (algebraic) edge[->] (solution_alg)
    (solution_alg) edge[->] (end);
  \end{tikzpicture}
\end{figure}

Veremos también que la transformada de Fourier es una herramienta básica en el análisis de señales aperiódicas que tienen energía finita. En este sentido, la transformada de Fourier juega el mismo papel que las series de Fourier para señales periódicas.

Antes de pasar a ver la definición de transformada de Fourier, vamos a definir el vocabulario que se utilizará para evitar ambigüedades.
\begin{definition}
  Llamamos frecuencia no angular de una función $f(t)$ a 
  $$
  f=\frac{1}{T}
  $$
  Que implícitamente define lo que llamamos período de $f(t)$:
  $$
  T=\frac{1}{f}
  $$

  Por otro lado, llamamos frecuencia angular (o solo frecuencia) al valor:
  $$
  \omega =2\pi \cdot f = \frac{2\pi}{T}
  $$
  Este valor es el que utilizaremos en su mayoría. En caso de precisar el valor de $f$, diremos \textit{frecuencia no angular}.
  \label{def:frecuencia}
\end{definition} 

\subsection{Definición}

Antes de dar la definición rigurosa de transformada de Fourier, vamos a motivar dicha definición a partir de las series de Fourier. Sea $f:\mathbb{R}\to\mathbb{C}$ una función que, en principio, consideraremos restringida a un intervalo finito de tiempo $T$. Podemos expresar $f(t)$ mediante su serie exponencial de Fourier:
\begin{equation}
    f(t)=\sum_{k=-\infty}^{\infty} C_k e^{j k \omega_0 t}
\end{equation}
donde $\omega_0 = \frac{2\pi}{T}$ es la frecuencia fundamental y los coeficientes vienen dados por:
\begin{equation}
    C_k = \frac{1}{T}\int_{-T/2}^{T/2}f(t) e^{-jk\omega_0 t} dt
    \label{eq:coeficientes_serie}
\end{equation}
Aquí, $f(t)$ es periódica con periodo $T$. Un detalle que no debe pasar por alto es que la función dada, $f(t)$, ya cuenta con sus propias características, esté o no representada por una serie. Si suponemos a esta presunta $f(t)$ periódica, entonces, ella tendrá su período $T$ y, por lo tanto, su frecuencia $\omega$. Por ejemplo $f(t)=\cos(t)$ tiene por sí misma $T=2\pi$ y $\omega=1$. Al representar una determinada $f(t)$ mediante una serie de Fourier, se está escribiendo esta misma función como una combinación lineal de senos y cosenos. 

Una representación gráfica hipotética de la amplitud de las frecuencias componentes de $f(t)$ podría ser la gráfica que se muestra en la figura \ref{fig:grafico_de_frecuencia}.
\begin{figure}[ht]
  \centering
  \begin{tikzpicture}
    \draw[->,gray] (-4.8,0) -- (4.8,0) node[right] {$k\omega$};
    \draw[->,gray] (0,-.1) -- (0,1.8) node[above] {$|C_k|$};

    \draw[red,very thick,fill] (0,0) node[below] {\color{black}\scriptsize$0$} -- (0,1) circle (2pt) node[above=3pt,fill=white,inner sep=0pt] {\scriptsize$\lvert C_0\rvert$};
    \foreach \k\i in {1/2,2/3,3/4,4/5} {
      \draw[red,very thick,fill] (\k,0) node[below] {\color{black}\scriptsize$\k\omega$} -- ($(\k,1/\i)$) circle (2pt) node[above] {\scriptsize$\lvert C_\k\rvert$};
      \draw[red,very thick,fill] (-\k,0) node[below] {\color{black}\scriptsize$-\k\omega$} -- ($(-\k,1/\i)$) circle (2pt) node[above] {\scriptsize$\lvert C_\k\rvert$};
    }
    \draw[dashed, smooth] plot coordinates {
      (-4,0.2) (-3,0.25) (-2,0.3333) (-1,0.5) (0,1)
      (1,0.5) (2,0.3333) (3,0.25) (4,0.2)
    };
  \end{tikzpicture}
  \caption{Representación de la amplitud de frecuencia discreta o de línea.}
  \label{fig:grafico_de_frecuencia}
\end{figure}
Esto nos sirve para definir algunos conceptos.
\begin{definition}[Curva envolvente]
  Llamamos envolvente a la curva suave que une cada punto $C_k$ en la gráfica de amplitud de frecuencias.
\end{definition}
\begin{definition}[Densidad espectral]
  Llamamos densidad espectral de frecuencia a la cantidad que nos dice que tan concentrada está la amplitud $C_k$ alrededor de la frecuencia $\omega_k$.
\end{definition}
Ahora, nuestro objetivo es analizar funciones \textbf{no periódicas} (o aperiódicas). Una función aperiódica puede interpretarse matemáticamente como una función periódica cuyo periodo tiende a infinito ($T \to \infty$).

Si observamos la ecuación \ref{eq:coeficientes_serie}, vemos que $C_k$ representa la amplitud de la frecuencia discreta $k\omega_0$. A medida que aumentamos el periodo $T$:
\begin{itemize}
    \item La frecuencia fundamental $\omega_0 = 2\pi/T$ disminuye, haciéndose cada vez más pequeña. Llamémosla $\Delta \omega$.
    \item Las líneas espectrales (los armónicos) en el gráfico de frecuencias se juntan cada vez más.
\end{itemize}

Para evitar que los coeficientes $C_k$ tiendan a cero cuando $T\to \infty$ (debido al factor $1/T$), multiplicamos la ecuación \ref{eq:coeficientes_serie} por $T$ para eliminar el factor de decaimiento:
$$
C_k \cdot T = \int_{-T/2}^{T/2} f(t) e^{-j(k\omega_0) t} dt
$$
Ahora, aplicamos el límite cuando $T \to \infty$. En este proceso ocurren tres cambios fundamentales:
\begin{enumerate}
    \item El intervalo de integración $[-T/2, T/2]$ se expande a toda la recta real $(-\infty, \infty)$.
    \item La frecuencia discreta $k\omega_0$ se convierte en una variable continua $\omega$, dado que el espaciado entre frecuencias $\Delta \omega \to 0$.
    \item La magnitud $C_k \cdot T$ se convierte en nuestra función transformada $F(\omega)$.
\end{enumerate}

De esta forma, llegamos a la definición formal de la Transformada de Fourier como una función integral de variable compleja.

\begin{definition}[Transformada de Fourier]
    Sea $f(t)$ una función tal que $\int_{-\infty}^{\infty} |f(t)| dt < \infty$. Se define la Transformada de Fourier de $f(t)$, denotada por $F(\omega)$ o $\mathcal{F}\{f(t)\}$, como:
    \begin{equation}
        F(\omega) = \int_{-\infty}^{\infty} f(t) e^{-j\omega t} dt
        \label{eq:transformada_definicion_formal}
    \end{equation}
    Esta función $F:\mathbb{R}\to\mathbb{C}$ representa la \textbf{densidad espectral} de la señal $f(t)$, indicando cómo se distribuye la amplitud y fase de la señal en el dominio continuo de la frecuencia.
\end{definition}

