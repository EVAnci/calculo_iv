\section{Resumen}

\begin{tcolorbox}[title=Serie de Taylor,resumen]
  Una función $f$ puede ser representada en serie de Taylor alrededor de un punto $z_0$ mediante 
  $$
  f(z)=\sum_{k=0}^\infty c_k(z-z_0)^k
  $$
  donde el coeficiente $c_k$ se calcula como
  $$
  c_k = \frac{f^{(k)}}{k!}=\frac{1}{j2\pi}\oint_\Gamma \frac{f(z)}{(z-z_0)^{k+1}}dz
  $$
  Siempre que $f$ sea analítica en el disco encerrado por la curva $\Gamma$.
\end{tcolorbox}

\begin{tcolorbox}[title=Serie de Laurent,resumen]
  Si una función no es analítica en $z_0$ el desarrollo en serie de Taylor no es válido. Entonces puede desarrollarse en serie de Laurent mediante 
  $$
  f(z)=\sum_{-\infty}^\infty c_k(z-z_0)^k
  $$
  donde $c_k$ se calcula como
  $$
  c_k = \frac{1}{j2\pi}\oint_\Gamma \frac{f(z)}{(z-z_0)^{k+1}}dz
  $$
  Esta serie puede escribirse en dos partes: la serie de Taylor, y la parte principal de la serie de Laurent, de modo que resulta
  $$
  f(z)=\sum_{k=0}^\infty c_k(z-z_0)^k +\sum_{k=1}^\infty \frac{b_k}{(z-z_0)^{k}}
  $$
  donde $c_k$ se calcula exactamente igual que antes, y $b_k$ se calcula como
  $$
  b_k=\frac{1}{j2\pi}\oint_\Gamma (z-z_0)^{k-1}f(z)dz
  $$
\end{tcolorbox}

\begin{tcolorbox}[title=Residuos,resumen]
  Se llama residuo al coeficiente $c_{-1}$ en la serie de Laurent:
  $$
  c_{-1}=\frac{1}{j2\pi}\oint_\Gamma f(z)dz
  $$

  \tcblower

  \textbf{Teorema del residuo}: Sea $\Gamma$ una trayectoria cerrada y $f$ analítica en $\Gamma$ y en su interior, menos en las singularidades $z_1,z_2,\dots,z_n$. Entonces 
  $$
  \oint_\Gamma f(z)dz = j2\pi\sum_{k=1}^n \res_{z=z_k}f(z)
  $$
\end{tcolorbox}

\begin{tcolorbox}[title=Evaluación de residuos,resumen]
  En general, sea $f$ una función con un polo de orden $m$ (puede ser orden 1 y ser simple). Entonces
  $$
  \res_{z=z_0}f(z)=\frac{1}{(m-1)!}~\lim_{z\to z_0}\left(\frac{d^{m-1}}{dz^{m-1}}[(z-z_0)^m f(z)]\right)
  $$

  \tcblower

  Únicamente para polos simples, para una función $f(z)=\frac{p(z)}{q(z)}$, el residuo puede calcularse como 
  $$
  \res_{z=z_0} f(z) = \frac{p(z_0)}{q'(z_0)}
  $$
\end{tcolorbox}
