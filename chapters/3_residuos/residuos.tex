\section{Residuos}

Un concepto muy importante que surge de la serie de Laurent es el de \textit{residuo}, con gran aplicación en el capítulo de transformada de Laplace y permite definir uno de los métodos más prácticos de antitransformación.

Para ver una conexión entre la serie de Laurent y la integral de una función, suponga que $f$ tiene un desarrollo de Laurent 
$$
f(z)=\sum_{-\infty}^\infty c_n(z-z_0)^n 
$$
en algún anillo $0<\lvert z-z_0\rvert <R$. Sea $\Gamma$ una trayectoria cerrada en este anillo que encierra a $z_0$. De acuerdo con el teorema \ref{teo:laurent} (serie de Laurent), los coeficientes de Laurent están dados por una fórmula integral. En particular el coeficiente $1/(z-z_0)$ es 
$$
c_{-1}=\frac{1}{j2\pi}\oint_\Gamma f(z)dz
$$
Por tanto,
$$
\oint_\Gamma f(z)dz=j2\pi (c_{-1})
$$
Si conoce este coeficiente en el desarrollo de Laurent puede obtener el valor de esta integral. Este hecho da una importancia especial a este coeficiente, de manera que tomará el nombre de \textit{residuo}. Como ya hemos visto en ejemplos anteriores, es posible conocer la serie de Laurent de una función sin tener que calcular los coeficientes. Entonces, puede encontrarse la serie de Laurent usando alguno de los métodos utilizados en los ejemplos a fin de evaluar la integral.

\begin{definition}[Residuo]
  Sea $f$ con una singularidad aislada en $z_0$ y desarrollo de Laurent $f(z)=\sum_{n=-\infty}^\infty c_n(z-z_0)^n$ en alguna corona $0<\lvert z-z_0\rvert < R$. Entonces el coeficiente $c_{-1}$ se llama \textbf{residuo} de $f$ en $z_0$ y se denota 
  $$
  \res_{z=z_0}f(z) \qquad \text{o bien} \qquad \text{Res}\left[f(z) \right\rvert_{z=z_0}
  $$
  o también puede encontrarse como $\text{Res}(f,z_0)$ en algunas bibliografías. Nosotros, preferentemente usaremos la primera notación ya que es más compacta.
\end{definition}

Esta idea puede extenderse para el caso en que $\Gamma$ encierra múltiples singularidades.

\begin{figure}[ht]
  \centering
  \begin{tikzpicture}[>=stealth]
    \begin{scope}[decoration={
        markings,
        mark=at position 0.3 with {\arrow{>}},
        mark=at position 0.6 with {\arrow{>}},
        mark=at position 0.9 with {\arrow{>} \node[below right] {$\Gamma$};},
      }]
      \draw[gray,very thick,rotate=20,postaction={decorate}] (0,0) ellipse (3 and 2);
    \end{scope}
    \begin{scope}[decoration={
        markings,
        mark=at position 0.1 with {\arrow{>} \node[right] {$\gamma_1$};},
      }]
      \draw[gray,postaction={decorate}] (1.5,0.8) circle (0.7);
    \end{scope}
    \draw[fill] (1.5,0.8) circle (2pt) node[left] {$z_1$};
    \begin{scope}[decoration={
        markings,
        mark=at position 0.1 with {\arrow{>} \node[right] {$\gamma_2$};},
      }]
      \draw[gray,postaction={decorate}] (0,1.2) circle (0.7);
    \end{scope}
    \draw[fill] (0,1.2) circle (2pt) node[left] {$z_2$};
    \begin{scope}[decoration={
        markings,
        mark=at position 0.1 with {\arrow{>} \node[right] {$\gamma_3$};},
      }]
      \draw[gray,postaction={decorate}] (0.8,-0.7) circle (0.7);
    \end{scope}
    \draw[fill] (0.8,-0.7) circle (2pt) node[left] {$z_3$};
    \begin{scope}[decoration={
        markings,
        mark=at position 0.1 with {\arrow{>} \node[right] {$\gamma_n$};},
      }]
      \draw[gray,postaction={decorate}] (-1.5,-0.8) circle (0.7);
    \end{scope}
    \draw[fill] (-1.5,-0.8) circle (2pt) node[left] {$z_n$};
  \end{tikzpicture}
  \caption{}
  \label{fig:residuos}
\end{figure}


\begin{theorem}[Teorema del residuo]
  Sea $\Gamma$ una trayectoria cerrada y $f$ diferenciable en $\Gamma$ y en todos los puntos encerrados por $\Gamma$, excepto para $z_1,~z_2,\dots,~z_n$, que son todas singularidades aisladas de $f$ encerradas por $\Gamma$ (ver figura \ref{fig:residuos}). Entonces
  $$
  \oint_\Gamma f(z)dz = j2\pi\sum_{k=1}^n \res_{z=z_k}f(z)
  $$
  En palabras, el valor de esta integral es $j2\pi$ veces la suma de los residuos de $f$ en las singularidades de $f$ encerradas por $\Gamma$. 
\end{theorem}

\begin{proof}
  Encerramos cada una de las singularidades $z_k$ en una trayectoria cerrada $\gamma_k$ que quede contenida dentro de $\Gamma$, que no encierre otras singularidades y que no interseque ninguna otra $\gamma_m$. Por el teorema de la deformación para dominios de conexidad superior (o segundo colorario de Cauchy, ecuación \ref{eq:dom_de_conex_sup}),
  $$
  \oint_\Gamma f(z)dz = \sum_{k=1}^n \oint_{\gamma_k}f(z)dz = j2\pi \sum_{k=1}^n \res_{z=z_k}f(z)
  $$
  quedando demostrado.
\end{proof}

El teorema del residuo es efectivo en la medida de nuestra eficiencia para evaluar los residuos de una función en sus singularidades. Si realmente tuviera que escribir el desarrollo de Laurent de $f$ alrededor de cada singularidad para mostrar el coeficiente del $1/(z-z_k)$ término, el teorema sería difícil de aplicar en muchos ejemplos. Lo que aumenta su importancia, es que, al menos para los polos, es una manera eficiente de calcular los residuos.

\subsection{Evaluación de los residuos}

Para que esta forma de resolver integrales pueda implementarse y resultar de utilidad, es necesario encontrar la manera de evaluar los residuos.

Tal como se verá, resulta más fácil evaluar y calcular un residuo que resolver una integral, por lo que este método constituye un aporte interesante en la resolución de integrales complejas.

Antes de proseguir con la integración, se pregunta lo siguiente. Para obtener un residuo, un simple coeficiente de una serie de Laurent ¿es necesario obtener toda la serie o existe una manera más económica? Para polos, la respuesta es que si existe una manera más económica. 

Entonces se deben distinguir las diferentes posibilidades que pueden presentarse para definir un residuo, a saber:
\begin{itemize}
  \item Residuo en un polo simple,
  \item Residuo en un polo múltiple,
  \item Residuo en el infinito.
\end{itemize}

\begin{theorem}[Residuo en un polo simple]
  Si $f$ tiene un polo simple en $z_0$. Entonces 
  $$
  \boxed{\res_{z=z_0} f(z) = \lim_{z\to z_0}(z-z_0)f(z)}
  $$
  es el residuo de $f$ en $z_0$.
\end{theorem}
\begin{proof}
  Si $f$ tiene un polo simple en $z_0$, entonces su desarrollo de Laurent alrededor de $z_0$ es 
  $$
  f(z)=\frac{c_{-1}}{z-z_0}+\sum_{k=0}^\infty c_k(z-z_0)^k
  $$
  ya que, es la serie de Taylor más la parte principal de la serie de Laurent, que, como tiene un polo simple, tiene un único término en alguna corona $0<\lvert z-z_0\rvert <R$. Entonces, si despejamos $c_{-1}$
  \begin{align*}
    f(z)&= \frac{c_{-1}}{z-z_0} + \sum_{n=0}^\infty c_k\frac{(z-z_0)^{k+1}}{(z-z_0)} \\ 
    f(z)&=\frac{1}{z-z_0}\left( c_{-1} + \sum_{n=0}^\infty c_k(z-z_0)^{k+1} \right) \\ 
    (z-z_0)f(z)&=c_{-1} + \sum_{n=0}^\infty c_k(z-z_0)^{k+1} 
  \end{align*}
  así, podemos hacer que la sumatoria converja a cero aplicando un límite 
  \begin{align*}
    \lim_{z\to z_0} (z-z_0)f(z) &= \lim_{z\to z_0} c_{-1}\sum_{n=0}^\infty c_k(z-z_0)^{k+1} \\ 
    \lim_{z\to z_0} (z-z_0)f(z) &= c_{-1} = \res_{z=z_0} f(z)
  \end{align*}
  quedando entonces la expresión que se quería demostrar.
\end{proof}

Otra fórmula, que algunas veces es más simple, para obtener el residuo en un polo simple se obtiene empezando con 
$$
f(z)=\frac{p(z)}{q(z)}
$$
con $p(z)$ y $q(z)$ analíticas, en donde se supone que $p(z_0)\neq 0$ y $q(z)$ tiene un cero simple en $z=z_0$ (de modo que, por el teorema \ref{teo:polos_y_ceros} de polos y ceros, tiene un polo simple en $z=z_0$, como se desea). Por la definición de cero simple, $q(z)$ tiene una serie de Taylor de la forma 
$$
q(z) = \sum_{k=1}^\infty q^{(k)}(z)(z-z_0)^k = q'(z_0)(z-z_0)+q''(z_0)\frac{(z-z_0)^2}{2!} + \dots
$$
Esta serie se sustituye en $f=p/q$ y después $f$ se sustituye en la fórmula de residuo
\begin{align*}
  \res_{z=z_0}&=\lim_{z\to z_0}(z-z_0)\frac{p(z)}{q(z)} \\ 
              &= \lim_{z\to z_0}\frac{(z-z_0)p(z)}{(z-z_0)\sum_{k=1}^\infty q^{(k)}(z)(z-z_0)^{k-1}} \\ 
              &= \lim_{z\to z_0}\frac{p(z)}{q'(z)}
\end{align*}
Pasando en limpio resulta 
$$
\boxed{\res_{z=z_0}f(z)=\res_{z=z_0}\frac{p(z)}{q(z)}=\frac{p(z_0)}{q'(z_0)}}
$$

Para terminar de afianzar los conceptos veamos un ejemplo.

\begin{example}
  Evaluar 
  $$
  \oint_\Gamma \frac{j4z-1}{\sin(z)}\,dz
  $$
  con $\Gamma$ la trayectoria cerrada de la figura \ref{fig:ej:polos}. 
  \begin{figure}[ht]
    \centering
    \begin{tikzpicture}[>=stealth]
      \draw[->,gray] (-1.5,0) -- (3.2,0);
      \draw[->,gray] (0,-2) -- (0,2);
      \foreach \i\a in {-1/-,1/,2/2,3/3} {
        \draw[gray] (\i,2pt) -- (\i,-2pt) node[below] {\scriptsize$\a\pi$};
      }
      \node[gray,below right] at (0,0) {\scriptsize$0$};
      \foreach \i in {-1,0,1,2} {
        \draw[fill] (\i,0) circle (1pt);
      }
      \begin{scope}[decoration={
          markings,
          mark=at position 0.1 with {\arrow{>}},
          mark=at position 0.4 with {\arrow{>}},
          mark=at position 0.7 with {\arrow{>}},
          mark=at position 0.9 with {\arrow{>} \node[above right] {\color{black}$\Gamma$};},
        }]
        \draw[gray,very thick,postaction={decorate}] (0,1.5) to [out=180,in=90] (-1.5,0.8)
          to [out=270,in=90] (-1.2,0.3)
          to [out=270,in=135] (-1.4,-1)
          to [out=315,in=180] (0,-0.5)
          to [out=0,in=180] (1.3,-2)
          to [out=0,in=240] (1.2,-0.5)
          to [out=60,in=270] (2.4,0)
          to [out=90,in=0] (0,1.5);
      \end{scope}
    \end{tikzpicture}
    \caption{$\Gamma$ encierra solamente las singularidades $-\pi,0,\pi,2\pi$ de $f$.}
    \label{fig:ej:polos}
  \end{figure}
  $\Gamma$ encierra los polos $-\pi,0,\pi,2\pi$ pero no otras singularidades de $f$. Por el teorema del residuo:
  \begin{equation}
    \oint_\Gamma \frac{j4z-1}{\sin(z)}dz = j2\pi\left(\res_{z=-\pi}f(z)+\res_{z=0}f(z)+\res_{z=\pi}f(z)+\res_{z=2\pi}f(z)\right)
    \label{eq:ej:polos}
  \end{equation}
  Aquí tenemos que 
  $$
  \res_{z=z_0}\frac{j4z-1}{\sin(z)}=\frac{j4z_0-1}{\cos(z_0)}
  $$
  Entonces, evaluando cada uno de los residuos en \eqref{eq:ej:polos}, resulta
  $$
  \oint_\Gamma \frac{j4z-1}{\sin(z)}dz = j2\pi \left( \cancel{(j4\pi-1) + \cancel{j40}-1 + \cancel{(-j4\pi+1)} + (-j8\pi+1) \right)
  $$
  reescribiendo todo, el resultado de la integral es 
  $$
  \oint_\Gamma \frac{j4z-1}{\sin(z)}dz = j2\pi (j8\pi-4) = -16\pi^2-j8\pi
  $$
\end{example}
