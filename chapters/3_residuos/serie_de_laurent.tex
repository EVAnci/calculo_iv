\subsection{Serie de Laurent}
\label{sec:serie_de_laurent}

En aplicaciones, a menudo es necesario desarrollar una función $f(z)$ alrededor de puntos en los cuales ya no es analítica, sino singular. Así, el teorema de Taylor deja de ser aplicable, por lo que se requiere un nuevo tipo de series, denominadas series de Laurent, que constan de potencias enteras positivas y negativas de $z-z_0$ y que son convergentes en alguna corona (acotada por dos círculos con centro en $z_0$) en donde $f(z)$ es analítica. La función $f(z)$ puede tener puntos singulares no sólo fuera del circulo más grande (como en las series de Taylor), sino también dentro del círculo más pequeño, lo cual es una nueva característica.

\begin{theorem}[Teorema de Laurent]
  Si $f(z)$ es analítica sobre dos círculos concéntricos\footnote{Recuerde que por la definición de analiticidad, esto significa que $f(z)$ es analítica en algún dominio que contiene a la corona, así como a sus círculos frontera.} $\Gamma_1$ y $\Gamma_2$ con centro en $z_0$ y en la corona entre éstos, entonces $f(z)$ puede representarse mediante la serie de Laurent
  \begin{equation}
    f(z)=\sum_{k=0}^\infty c_k(z-z_0)^k + \sum_{k=1}^\infty \frac{b_k}{(z-z_0)^k}
  \end{equation}
  donde la serie del primer término es la serie de Taylor y la serie del segundo término es la extensión de Laurent denominada \textbf{parte principal} de la serie de Laurent. Los coeficientes de esta serie de Laurent están definidos por las integrales
  $$
  c_k=\frac{1}{j2\pi}\oint_\Gamma \frac{f(z^\ast)}{(z^\ast-z_0)^{k+1}}dz^\ast \qquad b_k =\frac{1}{j2\pi}\oint_\Gamma (z^\ast - z_0)^{k-1}f(z^\ast)dz^\ast
  $$
  cada integral se toma en sentido antihorario alrededor de cualquier trayectoria cerrada $\Gamma$ que esté en la corona y abarque el circulo interior como se muestra en la figura \ref{fig:teorema_de_laurent}.
  
  Esta serie converge y representa a $f(z)$ en la corona abierta que se obtiene a partir de la corona dada al incrementar de manera continua el circulo $\Gamma_1$ y reducir $\Gamma_2$ hasta que ambos círculos llegan a un punto donde $f(z)$ es singular.
  \label{teo:laurent}
\end{theorem}

\begin{figure}[ht]
  \centering
  \begin{tikzpicture}[>=stealth]
    \draw[dashed,fill=gray!20,thick] (0,0) circle (2.1);
    \draw[red,fill=red!10,thick] (0,0) circle (2);
    \draw[red,thick,fill=gray!20] (0,0) circle (0.7);
    \draw[dashed,thick,fill=white] (0,0) circle (0.6);

    \path[red] (0,0) -- +(20:2) node[right] {$\Gamma_1$};
    \path[red] (0,0) -- +(280:0.7) node[below] {$\Gamma_2$};
    \draw[thick] (0,0) circle (2pt) node[right] {$z_0$};

    \begin{scope}[decoration={
        markings,
        mark=at position 0.1 with {\arrow{>}},
        mark=at position 0.1 with {\node[above right] {$\Gamma$};},
        mark=at position 0.6 with {\arrow{>}}
      }]
      \draw[thick,rotate=15,postaction={decorate}] (0.1,-0.2) ellipse (1.2 and 1.5);
    \end{scope}
  \end{tikzpicture}
  \caption{Teorema de Laurent}
  \label{fig:teorema_de_laurent}
\end{figure}

Aquí, en la figura \ref{fig:teorema_de_laurent} las líneas discontinuas indican el dominio donde $f(z)$ es analítica y continua. Como las curvas $\Gamma_1$ y $\Gamma_2$ están dentro de esa región, entonces también son continuas y $f$ es analítica sobre ellas.

\begin{proof}
  Sea $z$ cualquier punto en la corona dada (ver figura \ref{fig:teorema_de_laurent}). Entonces por la fórmula de la integral de Cauchy para dominios múltiplemente conexos (ecuación \eqref{eq:formula_de_la_integral_de_cauchy_para_dominio_multiplementeconexo_todo_antihorario}) se tiene  
  \begin{equation}
  f(z)=\frac{1}{j2\pi}\ointctrclockwise_{\Gamma_1}\frac{f(z^\ast)}{z^\ast-z}dz^\ast -\frac{1}{j2\pi}\ointctrclockwise_{\Gamma_2}\frac{f(z^\ast)}{z^\ast-z}dz^\ast 
  \label{eq:aplicación_ecuacion_cauchy_multiplemente_conexo}
  \end{equation}
  Cada una de las integrales anteriores se transforma como se hizo en la sección \ref{sec:serie_de_taylor}. La primera integral es precisamente como en la sección \ref{sec:serie_de_taylor}, por lo que se obtiene exactamente le mismo resultado
  \begin{equation*}
    \frac{1}{j2\pi}\ointctrclockwise_{\Gamma_1} \frac{f(z^\ast)}{z^\ast -z_0}dz^\ast = \sum_{k=0}^\infty c_k(z-z_0)^k
  \end{equation*}
  donde, al igual que en la sección \ref{sec:serie_de_taylor} el coeficiente $c_k$ es 
  $$
  c_k = \frac{1}{j2\pi}\oint_{\Gamma_1} \frac{f(z^\ast)}{(z^\ast-z_0)^{k+1}}dz^\ast
  $$
  Aquí es posible sustituir $\Gamma_1$ por $\Gamma$ por el teorema de la deformación \ref{teo:cauchy_doblemente_conexo}, ya que $z_0$, el punto donde el integrando de la integral no es analítico, no es un punto de la corona. Lo anterior demuestra la fórmula para el coeficiente $c_k$ ya que esto, al igual que en la sección \ref{sec:serie_de_taylor} es la expresión de la derivada $k$-ésima sobre el factorial de $k$
  $$
  c_k = \frac{f^{(k)}(z_0)}{k!}
  $$
  quedando demostrado el primer coeficiente.

  Ahora, para el segundo coeficiente, consideramos la segunda integral de \eqref{eq:aplicación_ecuacion_cauchy_multiplemente_conexo} y a partir de ella se obtendrá una fórmula para $b_k$. 

  Al ver la trayectoria $\Gamma_2$ el punto $z$ está por fuera de esta (según el teorema de las curvas de jordan (sección: \ref{sec:teorema_de_cauchy})). Ahora, a diferencia de la demostración que realizamos en la serie de Taylor, tenemos que la distancia del centro ($z_0$) al punto $z^\ast$ será siempre \textbf{menor} que la distancia del centro a $z$. Recordando que el punto $z^\ast$ está sobre la curva $\Gamma_2$ y el punto $z$ está en la corona. Entonces
  \begin{equation}
    \left\lvert \frac{z^\ast -z_0}{z-z_0} \right\rvert < 1
  \end{equation}
  por consiguiente, es necesario desarrollar $1/(z^\ast-z)$ en el integrando en potencias de $(z^\ast-z_0)/(z-z_0)$ (en vez del recíproco de esto) a fin de obtener una serie \textit{convergente}. En el integrando de la integral del segundo término de la ecuación \eqref{eq:aplicación_ecuacion_cauchy_multiplemente_conexo} se encuentra  
  $$
  \frac{1}{z^\ast-z} = \frac{1}{z^\ast -z_0 -(z-z_0)}
  $$
  Aquí, si sacamos factor común $-(z-z_0)$ obtenemos 
  $$
  \frac{-1}{(z-z_0)\left( 1-\frac{z^\ast-z_0}{z-z_0} \right)} = -\frac{1}{(z-z_0)}\cdot\frac{1}{\left(1-\frac{z^\ast-z_0}{z-z_0}\right)}
  $$
  Entonces de la misma forma que hicimos para demostrar la serie de Taylor, desarrollamos una serie geométrica, pero ahora buscamos una serie tal que su razón sea $q=\frac{z^\ast-z_0}{z-z_0}$. Tenemos entonces:
  $$
  \sum_{k=0}^\infty \left(\frac{z^\ast-z_0}{z-z_0}\right)^k = 1 + q + q^2 + \cdots = \frac{1}{1-q} \Rightarrow \frac{1}{z^\ast-z} = -\frac{1}{(z-z_0)}\cdot\sum_{k=0}^\infty q^k
  $$
  Reescribiendo esta última expresión en limpio y realizando algunas operaciones algebráicas resulta 
  $$
  \frac{1}{z^\ast-z} = -\frac{1}{(z-z_0)}\cdot\sum_{k=0}^\infty \left(\frac{z^\ast-z_0}{z-z_0}\right)^k = -\sum_{k=0}^\infty \frac{(z^\ast-z_0)^k}{(z-z_0)^{k+1}}
  $$
  Reemplazamos esta serie en la segunda integral de la ecuación \eqref{eq:aplicación_ecuacion_cauchy_multiplemente_conexo} resulta 
  \begin{gather*}
  -\frac{1}{j2\pi}\oint_{\Gamma_2} \frac{f(z^\ast)}{z^\ast-z}dz^\ast =
  \frac{1}{j2\pi}\oint_{\Gamma_2} f(z^\ast)\sum_{k=0}^\infty \frac{(z^\ast-z_0)^k}{(z-z_0)^{k+1}} \, dz^\ast 
  \end{gather*}
  Entonces, siguiendo el mismo razonamiento que antes, ya que la serie geométrica converge uniformemente, podemos usar reglas de diferenciación e integración, por tanto introducimos la integral dentro de la sumatoria
  $$
  \sum_{k=0}^\infty \frac{1}{j2\pi}\oint_{\Gamma_2}f(z^\ast)\frac{(z^\ast-z_0)^k}{(z-z_0)^{k+1}}\,dz^\ast
  $$
  Como el denominador no depende de la variable de integración ($z^\ast$), entonces podemos sacarlo de la integral 
  $$
  \sum_{k=0}^\infty \frac{1}{j2\pi(z-z_0)^{k+1}}\oint_{\Gamma_2}f(z^\ast)(z^\ast-z_0)^k\,dz^\ast
  $$
  Como esta integral se realiza dentro de la corona y $z_0$ no pertenece a la corona, entonces $f$ es analítica sobre la corona. Esto implica que podemos usar el teorema de la deformación (o primer colorario de Cauchy) e integrar sobre la curva $\Gamma$.

  Y por último, para acomodar el índice realizamos un desplazamiento de índice de la serie. En vez de comenzar en cero, comenzará en uno, y todos los índices dentro de la serie tendrán una unidad menos, siendo entonces 
  $$
  \sum_{k=1}^\infty \frac{1}{j2\pi(z-z_0)^k}\oint_{\Gamma} f(z^\ast)(z^\ast-z_0)^{k-1}\,dz^\ast
  $$
  Entonces llamamos $b_k$ a 
  $$
  b_k = \frac{1}{j2\pi}\oint_{\Gamma} f(z^\ast)(z^\ast-z_0)^{k-1}\,dz^\ast
  $$
  quedando la expresión final como 
  $$
  \sum_{k=1}^\infty \frac{b_k}{(z-z_0)^k}
  $$
  quedando demostrado el teorema.
\end{proof}

\subsection{Una expresión compacta para la serie de Laurent}

Claro está que la serie de Laurent es una generalización de la serie de Taylor. Véase que si la función $f(z)$ no es singular en $z_0$ (ni en ningún punto en el interior de $\Gamma_2$) entonces la serie de Laurent se reduce a la serie de Taylor con centro en $z=z_0$ analítico.

Si observamos la serie de Laurent con cuidado, vemos que son dos sumatorias. Estas sumatorias pueden reescribirse como una sola sumatoria desde menos infinito a más infinito (o para $k\in\mathbb{Z}$). Veamos, 
$$
f(z)=\sum_{k=0}^\infty c_k (z-z_0)^k + \sum_{k=1}^\infty b_k(z-z_0)^{-k}
$$
donde 
$$
c_k = \frac{1}{j2\pi}\oint_\Gamma \frac{f(z^\ast)}{(z^\ast-z_0)^{k+1}}\,dz^\ast \qquad b_k = \frac{1}{j2\pi}\oint_\Gamma \frac{f(z^\ast)}{(z^\ast-z_0)^{-k+1}}\,dz^\ast
$$
Entonces, en la parte principal de la serie de Laurent, vemos que tanto el coeficiente $b_k$ como $(z^\ast-z_0)$ tienen un exponente $k$ negativo. En esta circunstancia, podemos decir que el exponente es positivo, pero el intervalo es $(-\infty,1]$. Entonces como el primer término recorre el intervalo $[0,\infty)$ y el segundo término recorre $(-\infty,1]$ podemos escribir la serie de forma compacta como 
\begin{equation}
f(z)=\sum_{-\infty}^\infty c_k(z-z_0)^k
\label{eq:serie_de_laurent_compacta}
\end{equation}
donde $c_k$ es, como ya hemos visto anteriormente,
$$
c_k=\frac{1}{j2\pi}\oint_\Gamma \frac{f(z^\ast)}{(z^\ast-z_0)^{k+1}}\,dz^\ast
$$

Y, si recordamos, $z^\ast$ es la variable que hemos designado para recorrer la curva. Sin embargo, cuando realizamos la integral de línea, $z$ toma los valores de la curva misma, entonces este último coeficiente puede escribirse de una forma más práctica,
$$
c_k=\frac{1}{j2\pi}\oint_\Gamma \frac{f(z)}{(z-z_0)^{k+1}}\,dz
$$
siendo esta notación más cómoda para encontrar el término $c_k$, para un $k$ dado.

\subsection{Métodos prácticos para representar en serie de Laurent}

Así como para las series de Taylor, para obtener los coeficientes de la serie de Laurent en términos generales no se utilizan fórmulas de integración, sino varios métodos, algunos de los cuales se ilustrarán con ejemplos. Si mediante un procedimiento así se encuentra una serie de Laurent, entonces la unicidad garantiza que debe ser la serie de Laurent de la función dada en la corona dada. Es decir, cuando representamos una función $f(z)$ puede tener diferentes series de Laurent con el mismo centro, pero estas son en coronas distintas.

\begin{example}[Uso de la serie de Maclaurin]
  Encontrar la serie de Laurent de $z^{-5}\sin(z)$ con centro en $0$. 

  Sabemos que 
  $$
  \sin(z)=\sum_{n=0}^\infty \frac{(-1)^n}{(2n+1)!}z^{2n+1} 
  $$
  Podemos dividir esta serie por $z^5$, resultando
  $$
  z^{-5}\sin(z) = \sum_{n=0}^\infty \frac{(-1)^n}{(2n+1)!} \frac{z^{2n+1}}{z^5} = \boxed{\sum_{n=0}^\infty \frac{(-1)^n}{(2n+1)!} z^{2n-4}}
  $$
  Aquí la ``corona'' de convergencia es todo el plano complejo sin el origen.
\end{example}

\begin{example}[Sustitución]
  Encontrar la serie de Laurent de $z^2 e^{1/z}$ con centro en cero.

  Sabemos que 
  $$
  e^z = \sum_{n=0}^\infty \frac{z^n}{n!}
  $$
  Entonces 
  $$
  e^{1/z} = \sum_{n=0}^\infty \frac{1}{n!}\cdot\left(\frac{1}{z}\right)^n = \sum_{n=0}^\infty \frac{1}{n!\,z^n}
  $$
  Luego, multiplicando por $z^2$
  $$
  z^2 e^{1/z} = \sum_{n=0}^\infty \frac{z^2}{n!\,z^n} = \boxed{\sum_{n=0}^\infty \frac{1}{n!\,z^{n-2}}}
  $$
  donde $\lvert z \rvert > 0$.
\end{example}

\begin{example}[Desarrollo de series en coronas concéntricas diferentes]
  Encontrar todas las series de Laurent de 
  $$
  f(z)=\frac{1}{z^3-z^4}
  $$
  con centro en $0$. Nótese que si queremos encontrar el coeficiente $c_k$ usando la fórmula de Taylor es imposible. $f(0)$ y $f(1)$ no están definidos para este caso. Entonces, no es posible hacer un desarrollo en serie de Taylor de esta función centrada en $z_0=0$ debido a que las derivadas en cero no existen.
  
  Tenemos que, 
  $$
  \frac{1}{z^3-z^4} = \frac{1}{z^3}\cdot\frac{1}{1-z}
  $$
  Por tanto, como la serie de Laurent si es válida en este caso, podemos desarrollar por serie geométrica $1/(1-z)$ que converge si $0<\lvert z\rvert <1$, siendo 
  $$
  \frac{1}{z^3-z^4} = \frac{1}{z^3}\sum_{n=0}^\infty z^n = \sum_{n=0}^\infty z^{n-3}=\boxed{\sum_{k=-3}^\infty z^{k}}
  $$
  donde $z$ es válida en la corona $0<\lvert z \rvert < 1$. Aquí, vemos que $k$ toma tres valores negativos en la serie. Esto será importante ya que nos permite determinar que esta singularidad ($z_0=0$) es un polo múltiple de orden tres. 

  Sin embargo, también podemos desarrollar la misma $1/(1-z)$ en potencias negativas (que será válida en la corona $1<z$).
  $$
  \frac{1}{1-z}=\frac{-1}{z(1-z^{-1})}=-\sum_{n=0}^\infty \frac{1}{z^{n+1}}
  $$
  Entonces, siguiendo el mismo razonamiento que con la primer corona, tenemos
  $$
  \frac{1}{z^3-z^4}=-\sum_{n=0}^\infty \frac{1}{z^{n+1}}\cdot \frac{1}{z^3} = -\sum_{n=0}^\infty \frac{1}{z^{n+4}} = -\sum_{k=-4}^\infty z^{-k} = \boxed{-\sum_{m=-\infty}^4 z^{m}}
  $$
  donde $z$ es válida en la corona $\lvert z \rvert >1$. En este último caso se podría pensar que se ha llegado a una contradicción. ¿Infinitos términos negativos? El detalle es que, si bien la serie está centrada en $z_0=0$, es válida en la corona $\lvert z\rvert >1$ que \textbf{no incluye a} $z_0$. Por tanto no describe el comportamiento cerca de $z_0$, más bien, describe el comportamiento cerca de $z\to\infty$. 
\end{example}

Existen varios métodos más para evitar calcular las integrales y encontrar la ley de formación, sin embargo estos métodos abarcan los casos más generales y son suficientes para el objetivo del capítulo.

\subsection{Clasificación de singularidades}

Como bien ya vimos en la sección \ref{sec:singularidad} una singularidad son los puntos de una función donde esta no es diferenciable (y quizá ni siquiera este definida). La serie de Laurent nos permite conocer y clasificar mejor los distintos tipos de singularidades.

Si se desarrollan algunos pocos términos de la serie de Laurent, tal como está expresada en \eqref{eq:serie_de_laurent_compacta} desde un valor $-n$ hasta $n$ resultará
\begin{align*}
  f(z)=&\frac{c_{-n}}{(z-z_0)^n}+\frac{c_{-n+1}}{(z-z_0)^{n-1}}+\frac{c_{-n+2}}{(z-z_0)^{n-2}}+\cdots \\ 
       &\cdots + \frac{c_{-1}}{(z-z_0)} + c_0 + c_1(z-z_0)+c_2(z-z_0)^2 + \cdots \\ 
       &\cdots + c_{n-1}(z-z_0)^{n-1} + c_n(z-z_0)^n
\end{align*}

Entonces, a partir de esta expresión de la serie de Laurent definimos las siguientes singularidades.

\begin{definition}[Esencial]
  Las singularidades esenciales acontecen cuando el número de términos con exponente negativo (o la parte principal de la serie) en el desarrollo en serie de Laurent es infinito. En ese caso $z=z_0$ constituye una \textbf{singularidad esencial}.
\end{definition}
\begin{definition}[Polo]
  Los puntos singulares no esenciales, también llamados comúnmente polos, tienen lugar cuando el número de términos con exponente negativo del desarrollo de Laurent es finito. 

  Aquel número finito (llámese $m$) de términos negativos que tiene la serie de Laurent se denomina \textit{orden} del polo. Así entonces, sería un polo de orden $m$. O lo que es lo mismo, que $z=z_0$ es un polo múltiple de $f(z)$, cuyo orden de multiplicidad es $m$.
\end{definition}
\begin{definition}[Removible]
  Un punto $z_0$ es una singularidad removible de $f$ si $c_n=0$ para $n=-1,-2,\dots$. Es decir, no tiene términos negativos en la serie de Laurent.

  Este tipo de singularidades son de poco interés ya que la singularidad, como su nombre lo indica, puede quitarse.
\end{definition}

\begin{example}[Algunos ejemplos de singularidades]
  Un ejemplo de singularidad esencial es la función $f(z)=e^{1/z}$ que tiene una singularidad esencial en $z=0$. 

  Por otro lado, un ejemplo de polo múltiple es $f(z)=a/z^4$ que tiene un polo múltiple de orden $4$ en $z=0$. Para comprobar este caso, vamos a desarrollar su serie. En este caso, esta función tiene un desarrollo denominado desarrollo trivial, porque ya está en la forma de una serie de Laurent:
  $$
  \frac{1}{z^4}=\sum_{-\infty}^\infty c_nz^n
  $$
  donde $c_{-4}=1$ y todos los demás coeficientes $c_n=0$. Así la serie es simplemente 
  $$
  \frac{1}{z^4}
  $$
  Una vez desarrollada esta serie, multiplicamos por un escalar $a$ para obtener la función inicial.

  ¿Y por qué decimos que tiene un polo de multiplicidad cuatro? Porque el primer término negativo que tenemos es $c_{-4}$. Entonces, decimos que los términos componentes de la serie dados por los coeficientes $c_k$ son: 
  $$
  c_{-4}=1,~c_{-3}=0,~c_{-2}=0,~c_{-1}=0,~c_{0}=0,~c_1=0,\dots
  $$
  quedando así, definido el polo de orden cuatro.
\end{example}

A partir del ejemplo anterior, podemos enunciar un teorema bastante útil para analizar polos.

\begin{theorem}[Polos y ceros]
  Sea $f(z)$ analítica en $z=z_0$ tal que tiene un cero de $n$-ésimo orden en $z=z_0$. Entonces $g(z)=1/f(z)$ tiene un polo de $n$-ésimo orden en $z=z_0$.
  \label{teo:polos_y_ceros}
\end{theorem}
