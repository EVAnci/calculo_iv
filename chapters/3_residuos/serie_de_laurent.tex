\subsection{Serie de Laurent}

En aplicaciones, a menudo es necesario desarrollar una función $f(z)$ alrededor de puntos en los cuales ya no es analítica, sino singular. Así, el teorema de Taylor deja de ser aplicable, por lo que se requiere un nuevo tipo de series, denominadas series de Laurent, que constan de potencias enteras positivas y negativas de $z-z_0$ y que son convergentes en alguna corona (acotada por dos círculos con centro en $z_0$) en donde $f(z)$ es analítica. La función $f(z)$ puede tener puntos singulares no sólo fuera del circulo más grande (como en las series de Taylor), sino también dentro del círculo más pequeño, lo cual es una nueva característica.

\begin{theorem}[Teorema de Laurent]
  Si $f(z)$ es analítica sobre dos círculos concéntricos\footnote{Recuerde que por la definición de analiticidad, esto significa que $f(z)$ es analítica en algún dominio que contiene a la corona, así como a sus círculos frontera.} $\Gamma_1$ y $\Gamma_2$ con centro en $z_0$ y en la corona entre éstos, entonces $f(z)$ puede representarse mediante la serie de Laurent
  \begin{equation}
    f(z)=\sum_{k=0}^\infty c_k(z-z_0)^k + \sum_{k=1}^\infty \frac{b_k}{(z-z_0)^k}
  \end{equation}
  donde la serie del primer término es la serie de Taylor y la serie del segundo término es la extensión de Laurent. Los coeficientes de esta serie de Laurent están definidos por las integrales
  $$
  c_n=\frac{1}{j2\pi}\oint_\Gamma \frac{f(z^\ast)}{(z^\ast-z_0)^{n+1}}dz^\ast \qquad b_n =\frac{1}{j2\pi}\oint_\Gamma (z^\ast - z_0)^{n-1}f(z^\ast)dz^\ast
  $$
  cada integral se toma en sentido antihorario alrededor de cualquier trayectoria cerrada $\Gamma$ que esté en la corona y abarque el circulo interior como se muestra en la figura \ref{fig:teorema_de_laurent}.
  
  Esta serie converge y representa a $f(z)$ en la corona abierta que se obtiene a partir de la corona dada al incrementar de manera continua el circulo $\Gamma_1$ y reducir $\Gamma_2$ hasta que ambos círculos llegan a un punto donde $f(z)$ es singular.
\end{theorem}

\begin{figure}[ht]
  \centering
  \begin{tikzpicture}[>=stealth]
    \draw[red,fill=red!10,thick] (0,0) circle (2);
    \draw[red,thick,fill=white] (0,0) circle (0.6);

    \path[red] (0,0) -- +(20:2) node[right] {$\Gamma_1$};
    \path[red] (0,0) -- +(280:0.6) node[below] {$\Gamma_2$};
    \draw[thick] (0,0) circle (2pt) node[right] {$z_0$};

    \begin{scope}[decoration={
        markings,
        mark=at position 0.1 with {\arrow{>}},
        mark=at position 0.1 with {\node[above right] {$\Gamma$};},
        mark=at position 0.6 with {\arrow{>}}
      }]
      \draw[thick,rotate=15,postaction={decorate}] (0.1,-0.2) ellipse (1.2 and 1.5);
    \end{scope}
  \end{tikzpicture}
  \caption{Teorema de Laurent}
  \label{fig:teorema_de_laurent}
\end{figure}

\begin{proof}
  Sea $z$ cualquier punto en la corona dada. Entonces por la fórmula de la integral de Cauchy, se tiene que
  $$
  \frac{1}{j2\pi}\oint_{\Gamma_1}\frac{f(z^\ast)}{z^\ast-z}dz^\ast -\frac{1}{j2\pi}\oint_{\Gamma_2}\frac{f(z^\ast)}{z^\ast-z}dz^\ast 
  $$
  %pg 149
\end{proof}
