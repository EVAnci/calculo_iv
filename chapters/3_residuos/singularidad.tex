\subsection{Singularidad}

\begin{definition}
  Los puntos singulares de una función analítica $f(z)$ son puntos en los que $f(z)$ deja de ser analítica. Más precisamente, $z=c$ se denomina \textit{punto singular} de $f(z)$ si $f(z)$ no es diferenciable en $z=c$, aunque todo disco con centro en $c$ contiene puntos en lo que $f(z)$ es diferenciable. También se dice que $f(z)$ tiene una singularidad en $z=c$.
\end{definition}

\begin{example}[Algunos ejemplos sobre funciones con singularidades]
  La función 
  $$
  f(z)=\frac{1}{1-z}
  $$
  tiene una singularidad en $z=1$

  La función $\tan(z)$ tiene singularidades en $z=\pm \pi/2,\pm3\pi/2,\dots$.
\end{example}
