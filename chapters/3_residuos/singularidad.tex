\subsection{Singularidad}
\label{sec:singularidad}

\begin{definition}
  Los puntos singulares de una función analítica $f(z)$ son puntos en los que $f(z)$ deja de ser analítica. Más precisamente, $z=z_0$ se denomina \textit{punto singular} de $f(z)$ si $f(z)$ no es diferenciable (y quizá ni siquiera esté definida) en $z=z_0$, aunque toda vecindad de $z=z_0$ contiene puntos en los cuales $f(z)$ es analítica.
\end{definition}

\begin{definition}[Singularidad aislada]
  $z=z_0$ se denomina singularidad aislada de $f(z)$ si $z=z_0$ tiene una vecindad sin ninguna otra singularidad de $f(z)$.
\end{definition}

\begin{example}[Algunos ejemplos sobre funciones con singularidades]
  La función 
  $$
  f(z)=\frac{1}{1-z}
  $$
  tiene una singularidad en $z=1$

  La función $\tan(z)$ tiene singularidades aisladas en $z=\pm \pi/2,\pm3\pi/2,\dots$.
\end{example}
