\section{Fórmula de la integral de Cauchy}

La consecuencia más importante del teorema de la integral de Cauchy es la fórmula de la integral de Cauchy. Esta fórmula es de utilidad para calcular integrales. Igualmente importante es su rol primordial en la demostración del sorprendente hecho de que las funciones analíticas tienen derivadas de todos los órdenes (capítulo \ref{chpt:residuos}: residuos), y por tanto pueden ser representadas mediante series de Taylor. 

\begin{figure}[ht]
  \centering
  \begin{tikzpicture}
    \draw[dashed,rotate=20] (0,0) ellipse (3 and 2);
    \node at (2.3,1) {$D$};
    \begin{scope}[thick,decoration={
        markings,
        mark=at position 0.5 with {
          \coordinate (G) at (0,0);
          \node[left] at (0,0) {\footnotesize$\Gamma$}; 
        },
        mark=at position 0.6 with {\arrow{>}},
      }]
      \draw[red,postaction={decorate}] (0,0) ellipse (1.5 and 0.9);
      \draw[fill] (0.3,0.2) circle (1pt) node[right] {\footnotesize$z_0$};
    \end{scope}
  \end{tikzpicture}
  \caption{Fórmula integral de Cauchy.}
  \label{fig:f_in_cauchy}
\end{figure}

\begin{theorem}\label{teo:formula_de_la_integral_de_cauchy}
  Sea $f(z)$ analítica en un dominio simplemente conexo $D$. Entonces para cualquier punto $z_0$ en $D$ y cualquier trayectoria simple cerrada $\Gamma$ en $D$ que contenga a $z_0$ (figura \ref{fig:f_in_cauchy}) se cumple 
  \begin{equation}
    \oint_\Gamma \frac{f(z)}{z-z_0}dz = j2\pi f(z_0)
    \label{eq:formula_de_la_integral_de_cauchy}
  \end{equation}
  donde la curva $\Gamma$ se recorre en sentido antihorario.
\end{theorem}

\begin{proof}
  Por adición y sustracción, $f(z)=f(z_0)+[f(z)-f(z_0)]$. Al sustituir esto en el miembro izquierdo de la ecuación \eqref{eq:formula_de_la_integral_de_cauchy} y sacando de la integral el factor constante $f(z_0)$, se obtiene
  \begin{equation}
    \oint_\Gamma \frac{f(z)}{z-z_0}dz = f(z_0)\oint_\Gamma \frac{1}{z-z_0}dz + \oint_\Gamma \frac{f(z)-f(z_0)}{z-z_0}dz
    \label{eq:demostracion_formula_integral_cauchy}
  \end{equation}
  Resolviendo el primer término del miembro derecho, vemos que $\Gamma$ es una curva paramétrica no conocida que encierra al punto $z_0$. Con ello, podemos aplicar el teorema de deformación \ref{teo:cauchy_doblemente_conexo} y podemos sustituir la curva $\Gamma$ por una circunferencia centrada en $z_0$ de radio $r$, tal que $r>0$ pero lo suficientemente pequeño como para estar dentro de $D$. De este modo se cumple la siguiente igualdad
  \begin{gather*}
    f(z_0)\oint_\Gamma \frac{1}{z-z_0}dz = f(z)\oint_\gamma \frac{1}{z-z_0}dz
  \end{gather*}
  Y la curva $\gamma$ puede parametrizarse fácilmente como $\gamma(t)=z_0+re^{jt}$ para $t\in[0\leqslant t \leqslant 2\pi]$. Entonces la integral queda
  \begin{gather*}
    \oint_\gamma \frac{1}{z-z_0}dz = j\int_0^{2\pi}\frac{re^{jt}}{\cancel{z_0}+re^{jt}-\cancel{z_0}}dt = j\int_0^{2\pi}1 dt = j2\pi
  \end{gather*}
  por tanto la ecuación \eqref{eq:demostracion_formula_integral_cauchy} puede escribirse como
  \begin{gather*}
    \oint_\Gamma \frac{f(z)}{z-z_0}dz = j2\pi f(z_0) + \oint_\Gamma \frac{f(z)-f(z_0)}{z-z_0}dz
  \end{gather*}
  Siguiendo el mismo razonamiento para la integral del segundo término con $\gamma(t)=z_0+re^{jt}$
  \begin{align*}
    \int_0^{2\pi} \frac{f(\gamma(t))-f(z_0)}{e^{jt}} ~ je^{jt} ~dt &= j\int_0^{2\pi} f(\gamma(t))-f(z_0) dt \\
                                                                   &= j\left( \int_0^{2\pi} f(z_0+re^{jt}) dt - f(z_0)\int_0^{2\pi}1 dt \right)
  \end{align*}
  Aquí como la curva $\gamma$ es una circunferencia de radio $r$, podemos decir que $r\to 0$ ($r$ tiende a cero) sin modificar el valor de la integral ya que se cumple el teorema \ref{teo:cauchy_doblemente_conexo} de la deformación. Véase que si $r\to 0$ entonces $f(z_0+re^{jt})\to f(z_0)$, por tanto, para un $r\to 0$ podemos escribir 
  $$
   \int_0^{2\pi} f(z_0+re^{jt}) dt - f(z_0)\int_0^{2\pi}1 dt = f(z_0) \int_0^{2\pi}1dt - f(z_0)\int_0^{2\pi}1dt
  $$
  siendo entonces cero el único valor posible para la integral
  $$
  \oint_\Gamma \frac{f(z)-f(z_0)}{z-z_0}dz = 0
  $$
  Quedando entonces la ecuación \eqref{eq:demostracion_formula_integral_cauchy} como 
  $$
    \oint_\Gamma \frac{f(z)}{z-z_0}dz = j2\pi f(z_0) + 0
  $$
  lo que se quería demostrar.
\end{proof}

\subsection{Ejemplos}

El teorema \ref{teo:formula_de_la_integral_de_cauchy} nos brinda una forma muy sencilla de calcular integrales. Para entender la aplicación veamos algunos ejemplos.

\begin{example}
  Evaluar 
  $$
  \oint_\Gamma \frac{e^{z^2}}{z-j}dz
  $$
  para cualquier trayectoria cerrada que no pase por $j$.

  Tenemos dos posibles casos.
  \begin{itemize}
    \item \textbf{Caso I}: $\Gamma$ no encierra a $j$. En este caso la integral es cero por el teorema de Cauchy.
    \item \textbf{Caso II}: $\Gamma$ encierra a $j$. Por la fórmula de la integral de Cauchy, con $z_0=j$, 
      $$
      \oint_\Gamma \frac{e^{z^2}}{z-j}dz = j2\pi f(j) = j\frac{2\pi}{e}
      $$
  \end{itemize}
\end{example}

\begin{example}
  Evaluar 
  $$
  \int_\Gamma \frac{e^{2z}\sin(z^2)}{z-2}dz
  $$
  sobre cualquier trayectoria que no pase por 2.

  Al igual que antes, tenemos dos casos posibles.
  \begin{itemize}
    \item \textbf{Caso I}: Si $\Gamma$ no encierra a 2, entonces $f(z)/(z-2)$ es analítica y la integral es cero.
    \item \textbf{Caso II}: Si $\Gamma$ encierra a 2, usamos la fórmula de la integral de Cauchy:
      $$
      \int_\Gamma \frac{e^{2z}\sin(z^2)}{z-2}dz = j2\pi f(2) = j2\pi e^4 \sin(4)
      $$
  \end{itemize}
\end{example}
