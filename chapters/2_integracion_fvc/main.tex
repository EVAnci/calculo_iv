\chapter{Integración compleja}

Las funciones reales están definidas sobre conjuntos de números reales y con frecuencia se integran sobre intervalos. Las funciones complejas están definidas sobre conjuntos de puntos en el plano complejo y se integran sobre curvas. Antes de definir esta integral, es conveniente tener claro los conceptos de curvas paramétricas en el plano real y la integral de línea real (ver capítulo \ref{chpt:recordatorios}).

\section{Curvas en el plano}

Una curva en el plano complejo es una función $\Gamma :[a,b]\to\mathbb{C}$, definida en un intervalo real $[a,b]$ y que toma valores complejos\footnote{$\Gamma$: letra griega Gamma mayúscula.}. Para cada número $t$ en $[a,b]$, $\Gamma(t)$ es un número complejo, o un punto en el plano. El lugar geométrico de tales puntos es la gráfica de la curva. Sin embargo, la curva es más que un lugar geométrico de puntos en el plano. $\Gamma$ tiene una orientación natural, que es la dirección en la que el punto $\Gamma(t)$ se mueve a lo largo de la gráfica conforme $t$ crece de $a$ a $b$. En este sentido, es natural referirse a $\Gamma(a)$ como el \textit{punto inicial} de la curva y a $\Gamma(b)$ como el \textit{punto final}.

Si $\Gamma(t)=x(t)+jy(t)$, entonces la gráfica de $\Gamma$ es el lugar geométrico de los puntos $(x(t),y(t))$ para $a\leqslant t\leqslant b$. El punto inicial de $\Gamma$ es $(x(a),y(a))$ y el punto final es $(x(b),y(b))$ y $(x(t),y(t))$ se mueve del punto inicial al punto final conforme varía $t$ de $a$ a $b$. Las funciones $x(t)$ e $y(t)$ son todas las \textit{funciones coordenadas} de $\Gamma$.

\begin{example}
  Sea $\Gamma (t) = 2t + jt^2$ para $0\leqslant t\leqslant 2$. Entonces:
  $$
  \Gamma (t)=x(t)+jy(t),
  $$
  donde $x(t)=2t$ y $y(t)=t^2$. La gráfica de esta curva es la parte de la parábola $y=(x/2)^2$, que se muestra en la figura \ref{fig:ejemplo_parabola_gamma}.
  \begin{figure}[ht]
    \centering
    \begin{tikzpicture}
      \begin{axis}[
        axis lines = middle,
        xlabel = {$x$},
        ylabel = {$y$},
        xmin=-0.5, xmax=5.5,
        ymin=-0.5, ymax=5.5,
      ]
      
      % --- Curva Paramétrica ---
      \addplot [
        thick, 
        blue,
        domain=0:2,
        samples=100,
        ->,
        mark=none
      ]
      ({2*x}, {x*x});

      % --- Puntos Inicial y Final ---
      \node[label=left:{$\Gamma(0)$}] at (axis cs:0,0) {};
      \node[label=above right:{$\Gamma(2)$}] at (axis cs:4,4) {};

      % --- Líneas de guía punteadas ---
      \draw[dashed, gray] (axis cs:4,0) -- (axis cs:4,4);
      \draw[dashed, gray] (axis cs:0,4) -- (axis cs:4,4);

      \end{axis}
    \end{tikzpicture}
    \caption{Gráfica de la curva $\Gamma (t) = 2t + jt^2$, $0\leqslant t\leqslant 2$.}
    \label{fig:ejemplo_parabola_gamma}
  \end{figure}
  Conforme $t$ varía de $0$ a $2$, el punto $\Gamma (t)=(2t,t^2)$ se mueve a lo largo de esta gráfica del punto inicial $\Gamma(0)=(0,0)$ al punto final $\Gamma(2)=(4,4)$. Las flechas indican esta orientación.
\end{example}
