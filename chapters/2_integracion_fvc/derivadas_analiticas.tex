\section{Derivadas de funciones analíticas}

Ahora se demostrará que una función compleja holomorfa (o diferenciable) en $D$ como se mencionó en la sección \ref{sec:definicion_de_analiticidad} también es analítica en $D$. Entonces dicha función tiene derivadas de todos los órdenes en todos los puntos de $D$.

Para funciones reales no hay un resultado como este. Una función real diferenciable puede no tener segunda derivada. Y si tiene segunda derivada, puede no tener una tercera, y así sucesivamente. En otras palabras, si una función real es diferenciable una vez, nada puede concluirse acerca de la existencia de la segunda derivada o derivadas superiores. Así, en este sentido las funciones analíticas complejas se comportan mucho más sencillamente que las funciones reales.

\begin{theorem}[Derivadas de una función analítica]\label{teo:derivadas_de_todos_los_ordenes}
  Si $f(z)$ es analítica en un dominio $D$, entonces tiene derivadas de todos los órdenes en cada punto de $D$, las cuales entonces también son funciones analíticas en $D$.

  La derivada en un punto $z_0\in D$ se calcula como 
  \begin{equation}
    f^{(n)}(z_0) = \frac{n!}{j2\pi}\oint_\Gamma \frac{f(z)}{(z-z_0)^{n+1}}dz
  \label{eq:derivada_de_orden_n}
  \end{equation}
  con $n=1,2,\dots$ y $\Gamma$ una curva simple cerrada que encierra únicamente puntos de $D$ y que contiene a $z_0$. Aquí se asume que el sentido de integración es antihorario.
\end{theorem}

\subsection{Demostración}

Primero, vamos a demostrar que se cumple para la primer derivada, siendo la expresión
$$
f'(z_0) = \frac{1}{j2\pi} \oint_\Gamma \frac{f(z)}{(z-z_0)^2}dz
$$

\begin{proof}
  Para demostrar que la derivada en un punto $z_0\in D$ puede escribirse usando el teorema \ref{teo:derivadas_de_todos_los_ordenes}, partimos de la definición de derivada 
  $$
  f'(z_0) = \lim_{\Delta z \to 0} \frac{f(z_0+\Delta z)-f(z_0)}{\Delta z}
  $$
  Usando la fórmula de la integral de Cauchy (ecuación \eqref{eq:formula_de_la_integral_de_cauchy}), podemos escribir las funciones del numerador como
  \begin{align*}
    f(z_0+\Delta z) &= \frac{1}{j2\pi}\oint_\Gamma \frac{f(z)}{z-(z_0+\Delta z)} dz \\ 
    f(z_0) &= \frac{1}{j2\pi} \oint_\Gamma \frac{f(z)}{z-z_0}dz
  \end{align*}
  Y, reemplazando estas expresiones en el límite y operando algebraicamente:
  \begin{align*}
    f'(z_0)=&\lim_{\Delta z \to 0} \frac{1}{j2\pi\Delta z} 
      \left( 
        \oint_\Gamma \frac{f(z)}{z-(z_0+\Delta z)}dz - 
        \oint_\Gamma \frac{f(z)}{z-z_0}dz 
      \right) \\ 
    =&\lim_{\Delta z \to 0} \frac{1}{j2\pi\Delta z} 
      \left( 
        \oint_\Gamma \frac{f(z)}{z-z_0-\Delta z} - 
        \frac{f(z)}{z-z_0}dz 
      \right) \\ 
    =&\lim_{\Delta z \to 0} \frac{1}{j2\pi\Delta z} 
      \left( 
        \oint_\Gamma f(z)\frac{(z-z_0)-(z-z_0-\Delta z)}{(z-z_0-\Delta z)(z-z_0)}dz 
      \right) \\ 
    =&\lim_{\Delta z \to 0} \frac{1}{j2\pi\Delta z} 
      \left( 
        \oint_\Gamma f(z)\frac
          {\cancel{z}-\cancel{z_0}-\cancel{z}+\cancel{z_0}+f(z)\Delta z}
          {(z-z_0-\Delta z)(z-z_0)}dz 
      \right) \\ 
    =&\lim_{\Delta z \to 0} \frac{\cancel{\Delta z}}{j2\pi\cancel{\Delta z}} 
      \left( 
        \oint_\Gamma \frac{f(z)}{(z-z_0-\Delta z)(z-z_0)}dz 
      \right) \\  
    f'(z_0)=&\lim_{\Delta z \to 0} \frac{1}{j2\pi} 
      \left( 
        \oint_\Gamma \frac{f(z)}{(z-z_0-\Delta z)(z-z_0)}dz 
      \right) 
  \end{align*}
  Y, en esta última expresión, puede verse claramente que si $\Delta z\to 0$, entonces la integral tiende a 
  $$
  \boxed{f'(z)=\frac{1}{j2\pi}\oint_\Gamma \frac{f(z)}{(z-z_0)^2}dz}
  $$
  lo que se quería demostrar.
\end{proof}

Ahora, para demostrar que el teorema \ref{teo:derivadas_de_todos_los_ordenes} se cumple para todo $n$, vamos a suponer que la ecuación \eqref{eq:derivada_de_orden_n}:
\begin{equation*}
  f^{(n)}(z_0) = \frac{n!}{j2\pi}\oint_\Gamma \frac{f(z)}{(z-z_0)^{n+1}}dz
\end{equation*}
es cierta, y demostraremos que se cumple para $n+1$, quedando demostrada así, para todo $n=1,2,\dots$.

\begin{proof}
  Suponemos entonces que la ecuación \eqref{eq:derivada_de_orden_n} se cumple. Entonces
  \begin{equation}
    f^{(n+1)}(z_0) = \lim_{\Delta z\to 0} \frac{f^{(n)}(z_0+\Delta z)-f^{(n)}(z_0)}{\Delta z} 
    \label{eq:derivada_n_mas_uno}
  \end{equation}
  Por hipótesis, hemos dicho que \eqref{eq:derivada_de_orden_n} se cumple, por lo que podemos expresar las funciones del numerador de \eqref{eq:derivada_n_mas_uno} como 
  \begin{align*}
    f^{(n)}(z_0+\Delta z) &= 
      \frac{n!}{j2\pi}
      \oint_\Gamma \frac{f(z)}{(z-z_0-\Delta z)^{n+1}} dz \\ 
    f^{(n)}(z_0) &= 
      \frac{n!}{j2\pi}
      \oint_\Gamma \frac{f(z)}{(z-z_0)^{n+1}} dz 
  \end{align*}
  Reemplazando estas expresiones en el límite 
  \begin{align*}
    f^{(n+1)}(z_0) &= \lim_{\Delta z \to 0} \frac{n!}{j2\pi\Delta z} 
      \oint_\Gamma \frac{f(z)}{(z-z_0-\Delta z)^{n+1}}-
        \frac{f(z)}{(z-z_0)^{n+1}}dz \\ 
                   &= \lim_{\Delta z \to 0} \frac{n!}{j2\pi\Delta z} 
                   \oint_{\Gamma} f(z) \frac{(z-z_0)^{n+1}-(z-z_0-\Delta z)^{n+1}}
                   {(z-z_0-\Delta z)^{n+1}(z-z_0)^{n+1}}
  \end{align*}
  y de alguna manera se obtiene que
  $$
  (z-z_0)^{n+1}-(z-z_0-\Delta z)^{n+1} = \Delta z(n+1)(z-z_0)^n
  $$
  De forma tal, que al reemplazar en la integral resulta 
  $$
  f^{(n+1)}(z_0) = \lim_{\Delta z \to 0} 
    \frac{n!(n+1)\cancel{\Delta z}}{j2\pi\cancel{\Delta z}} 
    \oint_\Gamma \frac
      {f(z)\cancel{(z-z_0)^n}}
      {(z-z_0-\Delta z)^{n+1}(z-z_0)^{\cancel{n+1}}} ~ dz
  $$
  y tras aplicar el límite, al igual que para la primera derivada, cuando $\Delta z\to0$ resulta  
  $$
  f^{(n+1)}(z_0) = \frac{(n+1)!}{j2\pi} \oint_\Gamma 
  \frac{f(z)}{(z-z_0)^{n+2}}~ dz
  $$
  quedando así demostrado el teorema \ref{teo:derivadas_de_todos_los_ordenes}.
\end{proof}
