\section{La integral compleja}

\begin{definition}
Sea $f$ una función compleja y $\Gamma:[a,b]\to\mathbb{C}$ una curva suave en el plano. Suponiendo que $f$ es continua en todos los puntos de $\Gamma$, entonces la integral de $f$ sobre $\Gamma$ se define como
$$
\int_\Gamma f(z)\,dz = \int_a^b f(\Gamma(t))\Gamma'(t)\,dt
$$
Como $z=\Gamma(t)$ en la curva, esta integral se escribe frecuentemente como
\begin{equation}
\int_\Gamma f(z)\,dz = \int_a^b f(z(t))z'(t)\,dt
\label{eq:definicion_de_integral_compleja}
\end{equation}
Esta formulación tiene la ventaja de sugerir la manera que $\int_\Gamma f(z)dz$ es evaluada. Reemplazando $z$ con $z(t)$ en la curva y encontrando la expresión $dz=z'(t)dt$ sobre el intervalo $a\leqslant t\leqslant b$. 
\end{definition}

\begin{example}
  Evaluar $\int_\Gamma \bar{z} dz$ si $\Gamma(t)=e^{jt}$ para $0\leqslant t\leqslant \pi$.

  La gráfica de $\Gamma$ es la mitad superior del circulo unitario, orientado en sentido contrario al movimiento de las manecillas del reloj de $1$ a $-1$. En $\Gamma$, $z(t)=e^{jt}$ y $z'(t)=je^{jt}$. Más aún, $f(z(t))=\overline{z(t)}=e^{-jt}$ ya que $t$ es real. Entonces 
  $$
  \int_\Gamma f(z)dz=\int_0^\pi e^{-jt}\cdot je^{jt}dt=\int_0^\pi dt = \pi j
  $$
  \label{ej:integral_de_linea_en_complejos}
\end{example}

Es importante destacar que la integral del ejemplo \ref{ej:integral_de_linea_en_complejos} nos daba una curva paramétrica, esto no siempre es así. Aquí $\Gamma=e^{jt}$ con $0\leqslant t \leqslant \pi$, que es media circunferencia, pero podemos representar la \textbf{misma curva} con distintas parametrizaciones, y el resultado de la integral será el mismo. De hecho, todas las distintas parametrizaciones de media circunferencia con el mismo sentido de evolución que $\Gamma$ se denominan curvas equivalentes y el resultado de la integral es el mismo. Esto no quiere decir que el resultado de la integral sea igual para distintas curvas, sin embargo este tema lo trataremos más adelante.

\begin{example}
  Digamos que tomamos la curva $\Gamma:[0,\pi]\to \mathbb{C}$ tal que $\Gamma(t)=e^{jt}$. Una curva equivalente es $\Phi:[0,2\pi]\to\mathbb{C}$ tal que
  $$
  \Phi(t)=e^{jt/2}
  $$
  Véase que $\Phi$ representa media circunferencia, al igual que lo hace $\Gamma$, sin embargo su expresión paramétrica es diferente.\footnote{$\Phi$: Letra griega mayúscula ``Phi''}
\end{example}

Hasta aquí solo se ha integrado sobre curvas suaves. Se puede extender la definición a una integral sobre curvas suaves a trozos sumando las integrales sobre las componentes de la concatenación.

\begin{definition}
  Sea $\Gamma$ una concatenación de curvas suaves $\Gamma_1,\dots,\Gamma_n$. Sea $f$ continua en cada $\Gamma_k$. Entonces
  $$
  \int_\Gamma f(z)dz=\sum_{k=1}^n \int_{\Gamma_k} f(z)dz
  $$
\end{definition}

\subsection{La integral compleja en términos de integrales reales}

Es posible pensar en la integral de una función compleja sobre una curva como una suma de integrales de línea de funciones de valor real. Sea $f(z)=u(x,y)+jv(x,y)$ y, en la curva $\Gamma$, supongamos que $z(t)=x(t)+jy(t)$ para $t\in[a,b]$. Ahora
$$
f(z(t))=u(x(t),y(t)) + jv(x(t),y(t)) \qquad \text{y} \qquad z'(t)=x'(t)+jv'(t)
$$
entonces al hacer $f(z(t))\cdot z'(t)$, teniendo en cuenta la definición \ref{eq:definicion_de_integral_compleja},
\begin{align*}
  \int_\Gamma f(z)dz =& \int_a^b [u(x(t),y(t))x'(t)-v(x(t),y(t))y'(t)]dt + \\
                      & +j\int_a^b [v(x(t),y(t))x'(t)+u(x(t),y(t))y'(t)]dt
\end{align*}
En la notación de integrales de línea reales
\begin{equation}
  \int_\Gamma f(z)dz = \int_\Gamma u\,dx-v\,dy + j\int_\Gamma v\,dx+u\,dy
\end{equation}

Su forma de cálculo es similar a las integrales de línea reales que ya conocemos.

\subsection{Propiedades de las integrales complejas}

Cualquier cosa llamada integral, debe cumplir ciertas propiedades. A continuación vamos a definir las propiedades que cumplen las integrales complejas.

\subsubsection{Linealidad}

Sea $\Gamma$ una curva suave a trozos y sean $f$ y $g$ continuas en $\Gamma$. Sean $\alpha$ y $\beta$ números complejos. Entonces:
$$
\int_\Gamma (\alpha f(z)+\beta g(z))dz = \alpha \int_\Gamma f(z) dz + \beta \int_\Gamma g(z)dz
$$
El resultado puede extenderse a más sumas y constantes.

\subsubsection{Inversión de la orientación}

Sea $\Gamma:[a,b]\to\mathbb{C}$ una curva suave y $f$ continua en $\Gamma$. Entonces, una curva $\varPhi$ definida como
$$
\varPhi(t)=\Gamma(a+b-t) \qquad \text{para } a\leqslant t \leqslant b
$$
de modo que $\varPhi$ empieza donde termina $\Gamma$ ($\varPhi(a)=\Gamma(b)$) y termina donde $\Gamma$ empieza ($\varPhi(b)=\Gamma(a)$), es la curva $\Gamma$ recorrida en sentido inverso. Invertir la orientación cambia el signo de la integral
$$
\int_\Gamma f(z)dz=-\int_\varPhi f(z)dz
$$

\subsection{Independencia de la trayectoria}

Esto, podría haber sido un enunciado más de propiedades de integrales complejas. Sin embargo, no es una propiedad para todas las funciones complejas, sino para aquellas que sean \textbf{analíticas}. Esto es una versión compleja del teorema fundamental del cálculo para funciones analíticas. Establece que si $f$ tiene una antiderivada continua $F$, entonces el valor de $\int_\Gamma f(z)dz$ es el valor de $F$ en el punto final de $\Gamma$ menos el valor de $F$ en el punto inicial.

\begin{theorem}\label{teo:independencia_de_la_trayectoria}
  Sea $f$ continua y analítica en un conjunto abierto $G$ y suponemos que $F'(z)=f(z)$ para $z$ en $G$. Sea $\Gamma:[a,b]\to G$ una curva suave en $G$. Entonces
  $$
  \int_\Gamma f(z)dz = F(\Gamma(b))-F(\Gamma(a))
  $$
\end{theorem}

\begin{proof}
  Con $\Gamma(t)=z(t)=x(t)+jy(t)$ y $F(z)=U(x,y)+jV(x,y)$,
  \begin{align*}
    \int_\Gamma f(z)\,dz &= \int_a^b f(z(t))z'(t)\,dt = \int_a^b F'(z(t))z'(t)\,dt = \int_a^b \frac{d}{dt}F(z(t))\,dt \\
                         &= \int_a^b \frac{d}{dt} U(x(t),y(t))dt + j\int_a^b \frac{d}{dt}V(x(t),y(t))dt
  \end{align*}
  Ahora es factible aplicar el teorema fundamental del cálculo a las dos integrales reales de la derecha para obtener
  \begin{align*}
    \int_\Gamma f(z)\,dz &= U(x(b),y(b))+jV(x(b),y(b))-U(x(a),y(a))-jV(x(a),y(a)) \\ 
                         &= F(x(b),y(b))-F(x(a),y(a)) = \boxed{F(\Gamma(b))-F(\Gamma(a))}
  \end{align*}
\end{proof}

\begin{example}
  Calcular $\int_\Gamma (z^2+jz)dz$ con $\Gamma(t)=t^5-jt\cos(t)$ para $t\in[0,1]$.

  Este es un cálculo elemental, pero muy tedioso si se resuelve aplicando 
  $$
  \int_0^1 f(z(t))z'(t)dt 
  $$
  Sin embargo, si $G$ es todo el plano complejo, entonces $G$ es abierto, y $F(z)=z^3/3 +jz^2/2$ satisface $F'(z)=f(z)$. El punto inicial de $\Gamma$ es $\Gamma(0)=0$ y el punto final es $\Gamma(1)=1-j\cos(1)$. Por tanto,
  \begin{align*}
    \int_\Gamma (z^2 +jz)dz &= F(\Gamma(1))-F(\Gamma(0)) \\ 
                            &= F(1-j\cos(1))-\cancel{F(0)} \\ 
                            &= \frac{(1-j\cos(1))^3}{3}+\frac{j(1-j\cos(1))^2}{2} \\ 
                            &= (1-j\cos(1))^2 \left( \frac{1-j\cos(1)}{3} + \frac{j}{2} \right)
  \end{align*}
\end{example}

Una consecuencia del teorema \ref{teo:independencia_de_la_trayectoria} es que bajo las condiciones dadas, el valor de $\int_\Gamma f(z)\,dz$ depende solamente de los puntos inicial y final de la curva. Si $\varPhi$ es también una curva suave en $G$ teniendo el mismo punto inicial y final que $\Gamma$, entonces
$$
\int_\Gamma f(z)dz = \int_\varPhi f(z)dz
$$
Esto se llama \textbf{independencia de la trayectoria}.

Otra importante consecuencia del teorema es que si $\Gamma$ es una curva cerrada en $G$, entonces los puntos inicial y final coinciden y
$$
\oint_\Gamma f(z)dz = 0
$$
Consideraremos esta circunstancia con más detalle cuando veamos el \textit{teorema de Cauchy}. Aquí, el símbolo $\oint$ indica que la integral de línea se realiza sobre una curva \textit{cerrada}, de modo que no se deba aclarar que $\Gamma$ es una curva cerrada.

Ahora, veamos un ejemplo de función no analítica. 
\begin{example}
  Integrar $f(z) = \text{Re}(z)$ desde $0$ hasta $1+j$ a lo largo de $\Gamma_1$ y a lo largo de $\Gamma_2$ que consta de $\Gamma_{2a}$ y $\Gamma_{2b}$.
  \begin{figure}[ht]
    \centering
    \begin{tikzpicture}[scale=1.3]
      \draw[->,gray] (0,0) -- (2,0) node[right] {\scriptsize$x$};
      \draw[->,gray] (0,0) -- (0,2) node[right] {\scriptsize$y$};
      \draw[thin,gray] (1.6,-1pt) -- (1.6,1pt) node[below] {\scriptsize 1};
      \draw[thin,gray] (-1pt,1.6) -- (1pt,1.6) node[left] {\scriptsize 1};

      \draw[thick,dashed,blue!50!green] (0,0) -- (1.6,1.6) node[left=6pt] {\scriptsize$\Gamma_1$};
      \draw[thick,dashed,red] (0,0) -- (1.6,0) -- (1.6,1.6) node[below right] {\scriptsize$\Gamma_2$};
    \end{tikzpicture}
  \end{figure}
  Aquí $\Gamma_1$ puede representarse como $z(t)=t+jt$ para $t\in[0,1]$, de modo que
  \[
    z'(t) = 1+j \qquad \text{y}\qquad f(z(t)) = x(t) = t 
  \]
  Entonces, la integral queda
  \[
    \int_0^1 f(z(t))z'(t)dt = \int_0^1 t(1+j)dt = (1+j) \left[\frac{t^2}{2}\right\rvert_0^1 = \frac{1+j}{2} 
  \]
  Ahora, $\Gamma_{2a}$ puede representarse como $z(t)=t$ con $t\in[0,1]$,
  \[
    z'(t)=1 \qquad \text{y} \qquad f(z(t))=t
  \]
  Y por otro lado, $\Gamma_{2b}$ puede representarse como $z(t)=1+jt$ con $t\in[0,1]$,
  \[
    z'(t)=j \qquad \text{y} \qquad f(z(t))=1
  \]
  Entonces la integral sobre $\Gamma_2$ es 
  \[
    \int_{\Gamma_2} f(z)dz = \int_{\Gamma_{2a}} t\,dt + \int_{\Gamma_{2b}} j\,dt = \int_0^1 t \,dt + \int_0^1 j\,dt = \frac{1}{2}+j
  \]
  Con este ejemplo podemos concluir que para funciones \textbf{no analíticas} el resultado de la integral no solo depende de los puntos extremos de la trayectoria, sino también de la trayectoria misma.
\end{example}

\subsection{Integrales de series de funciones}

Es tentador pensar que para series infinitas se puede aplicar la siguiente propiedad:
\begin{quote}
  \centering
  ``La integral de una suma es la suma de las integrales''
\end{quote}
Sin embargo, como ya se ha visto en \textit{Cálculo III}, una serie infinita \textbf{no es una simple suma}. Esto implica que no siempre se puede intercambiar la integral y la suma. De hecho, en general, no es posible realizar dicho intercambio.

\begin{equation}
  \int_\Gamma \left(\sum_{n=1}^\infty f_n(z)dz \right) \stackrel{?}{=} \sum_{n=1}^\infty \int_\Gamma f_n(z)dz
\end{equation}

Lo bueno, es que si la serie es uniformemente convergente, entonces es posible realizar este tipo de intercambios.

\begin{theorem}[Integración término a término]
  Sea $\Gamma$ una curva suave y sea $f_n$ continua en $\Gamma$ para $n=1,2,\dots$. Suponemos que para cada entero positivo $n$ existe un número positivo $M_n$ tal que $\sum_{n=1}^\infty M_n$ converge y, para todo $z$ en $\Gamma$,
  $$
  \lvert f_n(z) \rvert \leqslant M_n
  $$

  Entonces $\sum_{n=1}^\infty f_n(z)$ converge absolutamente para todo $z$ en $\Gamma$. Más aún, si se denota $\sum_{n=1}^\infty f_n(z)=g(z)$, entonces
  $$
  \int_\Gamma g(z)dz = \sum_{n=1}^\infty \int_\Gamma f_n(z)dz
  $$
\end{theorem}

Este teorema es importante porque nos abre la puerta al teorema de Cauchy. Si bien no se demostrará, se explicará el concepto que cubre.

El teorema garantiza que, bajo ciertas hipótesis, la integral de una serie de funciones es igual a la serie de las integrales de cada función. Esto es útil porque en general no se puede intercambiar integrales y series sin verificar condiciones adicionales.

Antes de enunciar el teorema, se pide 
\begin{itemize}
  \item Que la curva $\Gamma$ sea suave, para que la integral esté bien definida.
  \item Que cada función $f_n$ sea continua sobre $\Gamma$.
  \item Que cada $f_n$ esté acotada uniformemente por $M_n$ en toda la curva. La serie \(\sum_{n=1}^\infty M_n\) converge (es decir, la suma de las \(M_n\) es finita).
\end{itemize}

Con esto, decimos que la serie \(\sum_{n=1}^\infty f_n(z)\) converge absolutamente para todo \(z\) en \(\Gamma\). Esto significa que la serie de los valores absolutos \(\sum_{n=1}^\infty |f_n(z)|\) converge, lo que implica que la serie original también converge. Entonces, si denotamos \(g(z) = \sum_{n=1}^\infty f_n(z)\), entonces:
\[
  \int_\Gamma g(z) \, dz = \sum_{n=1}^\infty \int_\Gamma f_n(z) \, dz
\]
Es decir, la integral de la suma es igual a la suma de las integrales.

La clave está en la convergencia uniforme de la serie \(\sum f_n(z)\). La condición \(|f_n(z)| \leq M_n\) con \(\sum M_n\) convergente es conocida como la prueba M de Weierstrass. Este prueba asegura que la serie \(\sum f_n(z)\) converge uniformemente en \(\Gamma\). Cuando una serie converge uniformemente a una función \(g(z)\) y cada \(f_n\) es continua, entonces \(g(z)\) es continua en \(\Gamma\) y se puede intercambiar la integral con la suma.
