\section{La integral compleja}

\begin{definition}
Sea $f$ una función compleja y $\Gamma:[a,b]\to\mathbb{C}$ una curva suave en el plano. Suponiendo que $f$ es continua en todos los puntos de $\Gamma$, entonces la integral de $f$ sobre $\Gamma$ se define como
$$
\int_\Gamma f(z)\,dz = \int_a^b f(\Gamma(t))\Gamma'(t)\,dt
$$
Como $z=\Gamma(t)$ en la curva, esta integral se escribe frecuentemente como
$$
\int_\Gamma f(z)\,dz = \int_a^b f(z(t))z'(t)\,dt
$$
Esta formulación tiene la ventaja de sugerir la manera que $\int_\Gamma f(z)dz$ es evaluada. Reemplazando $z$ con $z(t)$ en la curva y encontrando la expresión $dz=z'(t)dt$ sobre el intervalo $a\leqslant t\leqslant b$. 
\end{definition}

\begin{example}
  Evaluar $\int_\Gamma \bar{z} dz$ si $\Gamma(t)=e^{jt}$ para $0\leqslant t\leqslant \pi$.

  La gráfica de $\Gamma$ es la mitad superior del circulo unitario, orientado en sentido contrario al movimiento de las manecillas del reloj de 1 a -1. En $\Gamma$, $z(t)=e^{jt}$ y $z'(t)=je^{jt}$. Más aún, $f(z(t))=\overline{z(t)}=e^{-jt}$ ya que $t$ es real. Entonces 
  $$
  \int_\Gamma f(z)dz=\int_0^\pi e^{-jt}\cdot je^{jt}dt=\int_0^\pi dt = \pi j
  $$
\end{example}
