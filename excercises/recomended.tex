\chapter{Ejercicios Recomendados}

Esta sección contiene una serie de ejercicios generales, recomendados de cada capítulo. Hacer estos ejercicios puede servir para terminar de afianzar los conocimientos antes de rendir un examen.

Cada ejercicio se colocará con la respuesta, pero sin el procedimiento.

\section{Funciones de variable compleja}

\begin{exercise}
  Verificar la siguiente identidad:
  $$
  (\cos(\theta)+j\sin(\theta))^n = \cos(n\theta) + j\sin(n\theta)
  $$
  \textit{Solución}: Si se verifica, es llamada fórmula de Moivre.
\end{exercise}

\begin{exercise}
  Expresar las siguientes funciones de variable compleja en la forma $f(z)=u(x,y)+jv(x,y)$.
  \begin{enumerate}
    \item Para $f(z)=3z^{-2}+2z$. 

      \textit{Solución}:
      $$
      u(x,y) = \frac{3(x^2-y^2)}{(x^2+y^2)}+2x \qquad v(x,y)=2y-\frac{6xy}{(x^2+y^2)^2}
      $$
    \item Para $f(z)=\sin(2z)$.

      \textit{Solución}:
      $$
      u(x,y) = \sin(2x)\,\cosh(2y) \qquad v(x,y)=\cos(2x)\,\senoh(2y)
      $$
    \item Para $f(z)=\ln(z)$.

      \textit{Solución}:
      $$
      u(x,y) = \frac{1}{2}\ln(x^2+y^2) \qquad v(x,y)=\arctan\left(\frac{y}{x}\right)
      $$
  \end{enumerate}
  \label{ej:expresar_funciones_como_u_y_v}
\end{exercise}

\begin{exercise}
  Comprobar si las funciones del ejercicio \ref{ej:expresar_funciones_como_u_y_v} cumplen las ecuaciones de Cauchy-Riemann.

  \textit{Soluciones}:
  \begin{enumerate}
    \item Para $f(z)=3z^{-2}+2z$ las derivadas de $u$ y $v$ son:
      \begin{gather*}
        u_x = \frac{-6x^3+18y^2}{(x^2+y^2)^3}+2 \qquad v_y = u_x \\ 
        u_y = -\frac{18x^2y-6y^3}{(x^2+y^2)^3} \qquad v_x = -u_y
      \end{gather*}
      Por tanto, si cumple con las ecuaciones de Cauchy-Riemann.
    
    \item Para $f(z)=\sin(2z)$ las derivadas de $u$ y $v$ son:
      \begin{gather*}
        u_x = 2\cos(2x)\,\cosh(2y) \qquad v_y = u_x \\ 
        u_y = 2\sin(2x)\,\senoh(2y) \qquad v_x = -u_y
      \end{gather*}
      Por tanto, si cumple con las ecuaciones de Cauchy-Riemann.

    \item Para $f(z)=\ln(z)$ las derivadas de $u$ y $v$ son:
      \begin{gather*}
        u_x = \frac{x}{x^2+y^2} \qquad v_y = u_x \\ 
        y_y = \frac{y}{x^2+y^2} \qquad v_x = -u_y
      \end{gather*}
      Por tanto, si cumple con las ecuaciones de Cauchy-Riemann.
  \end{enumerate}
\end{exercise}

\begin{exercise}
  Hallar la función analítica $f(z)$ a partir de su parte real $u(x,y)=2e^x\cos(y)$ con la condición $f(0)=7$.

  \textit{Solución}: Integrando sobre las ecuaciones de Cauchy-Riemann obtenemos
  $$
  v(x,y) = 2e^x\sin(y) + C
  $$
  Para cumplir la condición $f(0)=7$ reemplazamos sobre $f(z)=2e^x\cos(y) + j (2e^x\sin(y) + C)$, para determinar $C$, resultando 
  $$
  C=-5j \quad \Rightarrow \quad f(z)=2e^x\cos(y) + j2e^x\sin(y) + 5
  $$
  Escrito en notación compacta:
  $$
  f(z)=2e^z + 5
  $$
\end{exercise}

\begin{exercise}
  Comprobar la función $u(x,y)=x^3-3xy^2$ es armónica y, en caso de ser afirmativo, calcular su armónica conjugada.

  \textit{Solución}: Si es armónica. Su armónica conjugada es 
  $$
  v(x,y)=3x^2y-y^3 +C
  $$
  Y, tomando una solución particular con $C=0$, componen la función:
  $$
  f(z)=x^3-3xy^2+j3x^2y - jy^3 \qquad \Rightarrow \qquad f(z)=z^3
  $$
\end{exercise}

\section{Integración Compleja}

\begin{exercise}
  Calcular las siguientes integrales utilizando las fórmulas de Cauchy.
  \begin{enumerate}
    \item Teniendo la curva $\Gamma$ dada por: 
      \begin{itemize}
        \item $\lvert z-1 \rvert =4$
        \item $\lvert z-2\rvert + \lvert z+2\rvert = 6$
      \end{itemize}
      Resolver: 
      $$
      \oint_\Gamma \frac{e^{3z}}{z-j\pi}dz
      $$
    \item Teniendo en cuenta $\Gamma$ dada por $\lvert z-1 \rvert =2$ resolver 
      $$
      \oint_\Gamma \frac{z}{2z-5}dz
      $$
  \end{enumerate}
  \textit{Soluciones}:
  \begin{enumerate}
    \item Para $\lvert z-1 \rvert =4$ tenemos 
      \begin{gather*}
        \text{Si } z = j\pi \quad \Rightarrow \quad \lvert j\pi - 1 \rvert < 4  
      \end{gather*}
      Entonces la singularidad del integrando queda encerrada en $\Gamma$. Usamos la fórmula de la integral de Cauchy:
      $$
      \oint_\Gamma \frac{e^{3z}}{z-j\pi}dz = j2\pi e^{-3j\pi} = \boxed{j\frac{2\pi}{e^{j3\pi}}}
      $$
      Para $\lvert z-2\rvert + \lvert z+2\rvert = 6$ representa una elipse como se muestra en la figura \ref{fig:ej:integral_elipse} tenemos que: 
      \begin{gather*}
        \text{Si }z=j\pi \quad \Rightarrow \quad \lvert j\pi-2\rvert + \lvert j\pi+2\rvert \approx 7.45 > 6 
      \end{gather*}
      \begin{figure}[ht]
        \centering
        \begin{tikzpicture}[scale=0.75,>=stealth]
          \draw[->,thin] (-4,0) -- (4,0) node[right] {$\Re(z)$};
          \draw[->,thin] (0,-3.5) -- (0,3.5) node[right] {$\Im(z)$};
          \begin{scope}[decoration={
              markings,
              mark=at position 0.1 with {\node[above right] {$\Gamma$};},
            }]
            \draw[very thick,red,postaction={decorate}] (0,0) ellipse (3 and 2.236);
          \end{scope}
          \foreach \x in {-3,-2,2,3} {
            \draw[thin] (\x,1pt) -- (\x,-1pt) node[below left] {\scriptsize$\x$};
          }
          \foreach \y\r in {-3.14/$-\pi$,-2.236/$-\sqrt{5}$,2.236/$\sqrt{5}$} {
            \draw[thin] (-1pt,\y) -- (1pt,\y) node[above right] {\scriptsize\r};
          }
          \draw[thick,fill=white] (0,-3.14) circle (2pt);
        \end{tikzpicture}
        \caption{Figura de la curva $\Gamma$ del ejercicio \ref{ej:integral_elipse}.}
        \label{fig:ej:integral_elipse}
      \end{figure}
      Por tanto queda fuera de la curva $\Gamma$ y 
      $$
      \oint_\Gamma \frac{e^{3z}}{z-j\pi}dz = 0
      $$
    \item Primero operamos sobre el denominador del integrando:
      $$
      2z-5 = 0 \quad\Rightarrow\quad z = \frac{5}{2}
      $$
      Para $\lvert z-1 \rvert =2$ tenemos que
      \begin{gather*}
        \text{Si } z=\frac{5}{2} \quad \Rightarrow \quad \lvert 5/2 -1 \rvert < 2
      \end{gather*}
      Por tanto está dentro de $\Gamma$. Usamos la fórmula de la integral de Cauchy:
      $$
      \oint_\Gamma \frac{z}{2z-5}dz = j5\pi 
      $$
  \end{enumerate}
  \label{ej:integral_elipse}
\end{exercise}
