\section*{Prefacio}

Este documento pretende ser una guía rigurosa basada en el programa de la materia \texttt{Cálculo IV} de la \href{https://um.edu.ar/}{Universidad de Mendoza}.

Los contenidos del programa de la materia incluyen 
\begin{itemize}
  \item Funciones de variable compleja,
  \item Ecuaciones de Cauchy-Riemann y analiticidad de funciones,
  \item Teorema de la integral de Cauchy,
  \item Serie de Taylor y Serie de Laurent,
  \item Teorema de los residuos para resolver integrales reales o complejas,
  \item Transformada de Fourier,
  \item Transformada de Laplace,
  \item Antitransformada de Laplace y resolución de ecuaciones diferenciales.
\end{itemize}

El principal motivo para hacer esta guía de estudio fué la idea de tener todos los temas tratados con el rigor y detenimiento de un libro, pero presentando enfasis fundamentalmente en los contenidos de la materia.


\subsection*{Importante}

Este documento ha sido escrito completamente usando \LaTeX, así lograr una calidad tipográfica buena a la par que práctica. Cualquier error, sugerencia o comentario que se desee realizar puede contactarme enviando un correo electrónico a \href{mailto:e.anci@alumno.um.edu.ar}{\texttt{e.anci@alumno.um.edu.ar}}.

Siempre antes de consultar por errores o sugerencias recomiendo descargar la última versión del documento \href{https://github.com/EVAnci/calculo_iv/releases}{aquí}.

Si deseas aportar o sumar contenido que te parezca relevante puedes hacerlo siguiendo las instrucciones en el \texttt{README.md} del siguiente repositorio: \href{https://github.com/EVAnci/calculo_iv}{\texttt{github.com/EVAnci/calculo\_iv}}.

\subsection*{Licencia}

Este documento está sujeto bajo la licencia \href{https://creativecommons.org/licenses/by-sa/4.0/}{\texttt{CC BY-SA}}. Esto significa que puedes compartir, adaptar, modificar e incluso vender el contenido que extraigas como desees siempre y cuando mantengas la licencia y atribuyas crédito al autor.

\begin{center}
\ccbysa
\end{center}
